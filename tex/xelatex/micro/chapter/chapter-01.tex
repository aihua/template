\chapter[导言]{导言}

\section{经济学与微观经济学}

什么是经济学?什么是微观经济学?它们之间有什么区别与联系?本节将就这些问题进行讨论。

\subsection{经济学}

经济学的定义是一个争议性较大的问题,迄今为止。并不存在一个被所有的经济学家都一致接受的有关经济学的定义。较为普遍而且为多数经济学家接受的定义是:经济学是研究个人、企业、政府以及其它组织如何在社会内进行选择,以及这些选择如何决定社会稀缺资源的使用的科学。该定义强调资源的稀缺性以及选择的重要性,不管是个人、企业,还是政府,或者是其它社会组织,所拥有的资源都是有限的,即资源是稀缺的。所谓资源的稀缺性(scarcity)是相对于人们的欲望而言的,人们的欲望是无穷无尽的,这些无穷无尽的欲望都要靠资源所生产的产品与劳务去满足。

除了时间与信息这两种重要的资源外,经济学中所讨论的资源基本上分为三种,即人力资源、自然资源与资本资人力资源是指能够并且愿意参与生产过程的人,自然资源是指处于自然状态的生产性资源,例如土地、矿藏等;资本资源是指在长期内能眵产生收入流(stream of income)的资本存量,资本资源简称为资本。

当资源被投入生产过程用以生产满足人们欲望的最终产品与劳务时,又称为生产要素。当我们讨论土地、资本、劳动等生产要素的有效利用时,实际上等同于讨论自然资源、资本资源、人力资源等资源的有效利用。

由于人们的欲望是无穷的,而用于生产满足人们各种各样欲望的资源是稀缺的,因此,每一时期人们都必须作出选择,以决定将稀缺的资源配置于哪一类产品与劳务的生产,满足人们哪一方面的欲望,由资源的稀缺性以及由此而限定的人类的选择引出了经济学中一个重要的概念:机会成本(opportunity cost)。使用一种资源的机会成本是指把该资派投入某一特定用途以后所放弃的在其它用途中所能获得的最大利益。只要资源是稀缺的,并且只要人们对于稀缺资源的使用进行选择,就必然会产生机会成本,某种资源一旦用于某种商品的生产就不能同时用于另一商品的生产,选择了一种机会就意味着放弃了另一种机会。由于个人、企业乃至一个国家所拥有的资源都是有限的,因此由个人、企业、国家所作出的选择都存在着机会成本。以一个国家的选择为例。假定一个国家欲将其既定的资源用于军用品与民用品两种产品的生产。在技术条件为不变的情况下,在既定的资源得到充分而又有效的利用的情况下,该国要想增加军用品的生产,就必须减少民用品的生产。增加一单位军用品生产的机会成本是所放弃的民用品生产的数量。一国所进行的这种选择以及这种选择所产生的机会成本由图1-1表示。

图1-1用横坐标表示军用品,用纵坐标表示民用品,以便分析军用品生产的机会成本。图中的曲线为{\hei 生产可能性边界线}(production possibility boundary),也称为{\hei 生产可能性曲线},或者称为{\hei 产品转换曲线}(product transformation curve),这条曲线表示一国在既定的资源与既定的技术条件下所能够生产的最大数量的军用品与民用品的组合。A点表示只生产民用品而不生产军用品时一国所能够生产的最大数量的民用品,B点表示只生产军用品而不生产民用品时一国所能够生产的最大数量的军用品。如果沿着生产可能性曲线从A点向B点移动,表示社会在增加军用品生产的同时将减少民用品的生产。生产可能性线凹向原点的特征表明从一种产品的生产转换为另一种产品的生产所产生的机会成本是递增的。例如,社会不断地将用于民用品生产的资源撤出转而用于军用品生产,随着军用品产量的不断增加而民用品产量的不断减少,每增加一单位军用品的生产,社会所放弃的民用品生产的数量越来越大。

个人与企业在既定的资源下进行选择时同样会产生机会成本。一个人将其既定的收入用于购买某种消费品或用于某一项投资,就不能将这笔收入用于购买另一种消费品或用于另一项投资,一个企业将其所拥有的既定资源用于某种商品的生产就不能同时将这些资源用于生产其它商品。机会成本产生于任一种选择行为。例如,人们对于时间的选择也存在着机会成本。一个人将某一段时间用于从事某一种活动,就不能同时将这一段时间用于另一活动。一个大学生大学毕业后选择读研究生就等于放弃了谋求正式职业的机会,读研究生的机会成本是他若不读研究生而工作所能够挣得的收入。

\subsection{微观经济学(microeconomics)}

微观经济学是经济学的一个重要分支。微观一词(micro)源于希腊文,意思为小。因此,有人把微观经济学叫做“小经济学”。微观经济学所研究的是个体经济单位的行为,例如研究消费者、企业的行为。微观经济学所涉及的是个体经济单位如何作出决策、以及这些决策受到哪些因素的影响。与微观经济学相对应的是经济学的另一重要分支宏观经济学(macroeconomics),宏观经济学又称为“大经济学”,研究的是总体经济行为与总量经济关系,例如国民生产总值、物价水平等。

微观经济学的中心问题是{\hei 价格问题}。这里的{\hei 价格}是指商品、劳务以及生产要素的相对价格,而不是绝对价格即物价水平,商品、劳务以及生产要素的相对价格调节着个体经济单位的经济行为,从而调节资源的配置与产品的产量。因此微观经济学又被许多经济学家称为价格理论。

对于{\hei 市场失灵}(market failure)情况下的资源配置效率损失以及如何解决市场失灵问题,微观经济学也将进行探讨。微观经济学对于所有权状况对资源配置的影响、激励机制的设计对于个体经济行为的影响同样进行理论上的探讨,建立在个人选择基础之上的公共选择也是微观经济学研究的重要内容。

\section{微观经济学中的实证分析与规范分析}

经济问题涉及到实证的问题与规范的问题这样两类问题,研究经济问题的经济学也就有实证经济学(positive economics)与规范经济学(normative economics)之分。与此相适应,对于微观经济问应存在着实证分析与规范分析的区别。

\subsection{实证经济学与规范经济学}

{\hei 实证经济学}所研究的是“是什么”的问题,或者说是人类所面临的实际经济问题“是如何解决”的问题。{\hei 规范经济学}所研究的是“应该是什么”的问题,或者说是人类所面临的实际经济问题“应该如何解决”的问题。

实证经济学不带有价值判断,而规范经济学则带有价值判断,实证经济学所表述的问题可以用事实、证据,或者从逻辑上加以证实或证伪。规范经济学所研究的问题则不能用事实、证据,或者从逻辑上加以证实或证伪。例如,分析政府对于某种产品征税而对该产品价格与产量的影响属于实证经济学研究的问题。人们可以利用事实或者从逻辑上证实或证伪自己所提出的命题。但是,分析政府对于某种产品征税的后果是好还是坏则属于规范经济学研究的问题。

\subsection{微观经济学中的实证分析与规范分析}

微观经济学既有实证分析又有规范分析,其中实证分析占重要的地位。{\hei 微观经济学}是描述个体经济单位如何进行选择的科学。通过描述个体经济单位的选择行为,微观经济学揭示了经济运行状况。要通过描述个体经济单位的行为揭示经济的运行状况,就必须对个体经济单位的行为作出若干假定,例如,假定个体经济单位的行为是理性的,即企业追求利润的最大化,消费者追求效用的最大化等等。在对个体经济单位的行为作出若干假定的基础上,对影响个体经济单位行为的因素加以解释、说明或者作出预测。这种分析不带有任何价值判断,因而属于实证分析观经济学中有关消费者行为、企业行为、市场定价行为、公共选择行为、政府干预行为,以及这些行为所导致的结果的分析都属于实证分析。通过对个体经济单位进行实证分析可以说明,一旦影响个体经济单位行为的因素发生变化。个体经济单位的行为以及受其行为影响的经济量将发生什么样的变化。例如,某种资源的短缺而造成的该资源较高的价格将会提高用这件资源所生产的产品的成本,从而使这类产品的价格提高。

实证分析也可以揭示政府微观经济政策的后果。例如,为了保证公路的正常养护与维持公路的正常运营,政府可以征收养路费,也可以征收燃油税补贴公路部门,政府不同的经济政策所产生的后果是不同的,养路费是每年对车主征收一笔固定的费用。车主所交的燃油税则与他所使用的油量成比例。不同的付费方式会异致消费者不同的行为,如果只收养路费而不征燃油税。车主使用公路的频率将会提高。如果只征收燃油税而不收养路费,车主使用公路的频率比起只交养路费而不交燃油税情况下使用公路的频率要低。不仅如此,是征燃油税还是收养路费对汽车与汽油的需求的影响是不同的。征收较高的燃油税显然会提高汽油的价格,降低对于汽油的需求量。征收较高的养路费则会使那些汽车的使用频率不高的人减少对于汽车的需求,微观经济学也涉及到规范分析,例如,经济运行最优状态的确定涉及价值判断。究竟什么样的状态才是经济运行的最优状态,不同的人有不同的看法。经济运行的结果是好还是坏,经济运行究竟应该达到什么样的结果才是好的,往往难以取得一致的意见。经济学家在实证性问题上容易达成共识,但是在规范性问题上却分歧较大。例如,任一位经济学家都不会否认,在市场价格机制完善的经济中,在其它条件给定的情况下,提高燃油税会减少对于汽油的需求。但是,应不应该提高燃油税,提高燃油税的结采是好还是坏,则可能是仁者见仁,智者见智。

实证分析与规范分析较之定性分析与定量分析是两类不同的范畴,不应将这两类范畴相混淆。定性分析并不一定带有价值判断,因此定性分析可能是一种实证分析。例如,政府最低工资的政策是否会减少就业量这是一种定性的分析。如果进一步研究政府最低工资政策使就业量减少了多少则是定量的分析。这要通过统计学或计量经济学方法加以分析。这两种方法都属于实证分析。因为用这两种方法所分析的问题都可以用事实加以证实或证伪。

微观经济学的主要内容是进行实证分析,但是对若干问题也进行规范分析。

\section{微观经济学的功能与用途}

\subsection{微观经济学的功能}

由于微观经济学的主要内容是进行实证分析,因此它具有两个重要的功能:解释功能与预测功能。即它可以用于解释或预测微观经济行为主体。例如,生产者或消费者的经济行为。

微观经济学对于微观经济行为主体经济行为的解释或预测建立在理论的基础上。像其它任何一门科学理论一样,经济学理论依靠一套基本概念、假设以及公理化的推理体系对于所观察到的经济现象进行解释。经济学中的一些基本概念例如效用、利润、成本、价格、供给、需求等等都有明确的定义。在这些概念的基础上,经济学对经济主体的行为作出若干假设,通过假设与推理对经济现象进行解释。例如,在微观经济学的消费者理论中,假设消费者行为目标是追求效用的最大化。利用这一假设,该理论解释了消费者如何进行时间分配,以便通过收入与闲暇的最优选择达到效用的最大化;或者消费者如何进行自己所拥有的收入与财产的分配,消费者购买什么样的商品、保留多少储蓄、选择什么样的资产才能使效用达到最大化。该理论还解释了工资、利率或消费品价格的变化如何影响消费者的效用最大化行为。又如,在微观经济学的厂商理论中,假设厂商的行为目标是追求利润的最大化。利用这一假设,该理论解释了产品与生产要素的投入决策,以达到利润的敢大化。该理论还解释了产品与生产要素价格的变化如何影响厂商的利润最大化行为。

经济理论也是进行经济预测的基础。一旦对经济行为主体的行为作出某种假设,我们就可以对若干经济量进行预测。例如,如果假设厂商的目标是追求利润的最大化,就可以预测,在其它条件不变的情况下,厂商所使用的投入物价格的上升,比如劳动者工资率的提高、原材料价格的上涨等将会使厂商减少产量。反之,若厂商所使用的投入物价格下降将会使厂商增加产量。

利用统计学与计量经济学技术可以对经济理论加以模型化,从而可以对所观察到的经济现象作出比较精确的解释,或对经济量作出比较精确的预测。经济镆型是对经济理论的数学表述。一旦利用统计学与计量经济学的技术建立起经济模型,便可以将对于经济主体经济行为的定性分析与预测变为定量的分析与预测。例如,我们可以预测,在其它条件不变的情况下,若原材料的价格上升10\%,厂商的产量将会下降多大的幅度。

无论是对于所观察的经济现象的解释,还是对于经济主体行为的预测,都离不开和经济量打交道。实证经济学所涉及的基本上是经济量之间的关系,例如投入与产出之间的关系、成本与利润之间的关系、价格与数量之间的关系等等。在经济理论或经济模型中,为了分析的方便,根据分析的需要,对经济量的特征作若干规定。例如,把经济量分为常量(constant)与变量(variable);内生变量(endogenous variable)与外生变量(exogenous variable);存量(stock variable)与流量(flow variable)等。

{\hei 常量}是在某段时间、某种情况下不发生变化的变量是在某段时间、某种情况下会发生变化的量。

{\hei 内生变量}是在一个体系内或在一个模型中可以得到说明的变最;外生变量是在一个体系内或在一个模型中不能得到说明的变量。例如,在某种商品比如小麦均衡价格决定的模型中,小麦的供给量与需求量是内生变量,可以在方程体系内得到解释,并可以进行求解;在该方程体系中,其它商品的价格则是外生变量。不能在该方程体系中得到解释和求解。一个变量究竞是内生变量还是外生变量视它在方程中的作用而定。在一个方程体系中是内生变量的量,在另一个方程体系中有可能成为外生变而在一个方程体系中是外生变量的量,在另一个方程体系中有可能成为内生变量。例如,在其它商品均衡价格决定的模型中,小麦的产量与价格都可能成为外生变在一个一般均衡体系中,所有商品的价格与产量都是内生变量。

{\hei 存量}是在某一个时点上发生的量值,例如在某一时点上某商店所库存的彩电量,某一企业所拥有的资本量,某一国家所拥有的人口量等。{\hei 流量}是在某一定时期内发生的量值,例如某商店在某个月内卖出的彩电量,某一企业在一年内的投资量,一国一年内的人口出生数等。

在微观经济学中,利用经济理论或经济模型对经济量进行分析、对经济现象进行解释、对经济主体的经济行为及其结果进行预测并不是完美无缺的。由于经济学是一门社会科学,它所描述的经济变量之间的关系非常复杂,因此,经济学不可能像物理学那样对变量之间的关系描述得非常精确。在利用经济模型对经济量进行预测时更难达到较高的准确性。这表明经济学的定性分析与定量分析方法都有待于进一步完善。

\subsection{微观经济学的用途}

无论从理论上讲还是从实践上讲,微观经济学都具有广泛的用途。

从理论上讲,它既是经济学理论研究的一个重要的领域,又是经济学其它研究领域的基础。宏观经济学、国际经济学、比较经济学、发展经济学、财政学等研究领域需要微观经济学理论作为基础。微观经济学理论甚至被引入政治学、社会学的研究领域中,用于研究人的政治决策行为与社会决策行为,从实践上讲,微观经济学理论既可以用于指导企业或消费者的决策行为,又可以用于指导政府的决策行为。

对于消费者来说,微观经济学中资产组合的理论、消费与储蓄的时际选择理论等可以用于指导他们的选择行为。

对于企业而言,微观经济学中的成本与收益分析、厂商定价理论、资本与投资理论等可以用于指导其投资决策与生产、销售决策,微观经济学中的机会成本的分析比会计成本的分析对于企业经理人员的决策更具有指导意义。

微观经济学理论不仅仅对消费者、企业这样的微观经济主体的决策行为具有指导作用,对于政府的决策行为同样具有指导作用。政府的许多决策如果能够科学地运用微观经济学知识,就会减少决策的盲目性。例如,若政府试图通过限制某种商品的价格而增加消费者的福利,政府必须了解该商品需求的价格弹性。如果该商品需求的价格弹性很小,就不应限制该商品的价格。否则,限价的结果只会减少而不是增加消费者的福利。政府若举办某项公共工程,也应该运用微观经济学中的成本效益分析理论对欲举办的公共工程进行经济分析,以便减少其决策的盲目性。


\begin{enumerate}
\item 微观经济学研究哪些问题?

\item 实证经济学与规范经济学有什么区别?

\item 微观经济学有什么功能与用途?

\item 定性分析与定量分析有什么区别与联系?

\item 机会成本与资源的稀缺性之间有什么联系?

\item 存量与流量间的区别与联系是什么?
\end{enumerate}

\newpage
