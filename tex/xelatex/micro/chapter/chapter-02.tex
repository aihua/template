\chapter[需求、供给与局部均衡价格的决定]{需求、供给与局部均衡价格的决定}

需求与供给理论是微观经济学中的基本理论,不了解需求与供给就不了解经济学。微观经济学的中心问题是价格理论。价格决定于供给与需求。无论是商品与劳务的价格,还是生产要素的价格,或者是资金的价格,都由其供给与需求决定。因此,不研究供给与需求,就无法探讨微观经济学问微观经济学关于价格理论的研究分为局部均衡与一般均衡两方面展开。局部均衡理论所探讨的是单个商品、劳务或生产要素价格与数量的决定。一般均衡理论所探讨的是所有商品、劳务,以及生产要素价格与数量同时决定的情况。本章所探讨的是局部均衡价格决定的问题,即探讨单个商品的价格是如何决定的。

\section{需求}

关于需求问题我们分个人需求与市场需求两个方面进行讨论。

\subsection{个人需求}

个人需求是一个表列,它表示一个人在某一特定时间内,在各种可能的价格下,他将购买的某种商品的各种数量。个人对于商品的需求需要具备两个条件:第一,个人具有购买意愿;第二,个人具有支付能力。没有支付能力的购买意愿只是自然需要而不构成需求。个人需求受许多因素的影响,主要因素有:个人的偏好、个人的资产与收入、个人所购买的商品的价格、与个人所购买的商品的价格有关的其他商品的价格。消费者对商品未来价格的预期等。如果用需求函数表示个人对于某一种商品的需求,个人的需求函数可以表示为:

\begin{equation}\label{eq:2-1}
	q_x^d=f(P_x;P_1,P_2,\dots,P_n;P_e;M;h)
\end{equation}

其中,$q_x^d$表示消费者对$x$商品的需求。$P_x$表示$x$商品的价格。$P_i(i=1,2,\dots,n)$表示第$i$种商品的价格。$P_e$表示消费者对于未来价格的预期。$M$表示消费者的收入,$h$表示消费者的偏好。

\eqref{eq:2-1}式是一个多元函数。研究多元函数比较复杂,而且对于探讨局部均衡价格的决定来说,没有必要讨论复杂的多元函数,因此,只研究比较简单的一元函数。通常的一元需求函数有两种形式:一种形式是讨论的商品的需求量与它自身的价格之间的关系;另一种形式是讨论商品的需求量与消费者收入之间的关系。由于本章的局部均衡价格理论主要是探讨商品均衡价格与均衡数量的决定,因此,我们重点讨论描述商品的需求量与其自身价格之间关系的一元需求函数,而假定影响消费者需求的其他条件不变。

如果假定其他条件不变,而只有消费者所购买的商品的价格发生变化,那么一元需求函数可以表示为:

\begin{equation}\label{eq:2-2}
	q_x^d=a-bp_x
\end{equation}

或

\begin{equation}\label{eq:2-3}
	q_x^d=ap_x^{-\alpha}
\end{equation}

或者其他形式。其中\eqref{eq:2-2}式是线性需求函数,\eqref{eq:2-3}式是非线性需求函数。\eqref{eq:2-2}式与\eqref{eq:2-3}式中的$a$、$b$、$\alpha$为参数。根据需求函数,我们可以画出需求表或需求曲线,以线性需求函数为例,假定某消费者的需求函数为:


\begin{equation}\label{eq:2-4}
	q_x^d=8-p_x
\end{equation}

与\eqref{eq:2-4}式的需求函数相对应的需求表为:

\begin{table}
	\caption{Line}
	\centering
	\begin{tabular}{|r|r|r|r|r|r|r|r|r|}
		\hline
		$p_x$ 	& 8 & 7 & 6 & 5 & 4 & 3 & 2 & 1 \\
		\hline
		$q_x^d$ & 0 & 1 & 2 & 3 & 4 & 5 & 6 & 7 \\
		\hline
	\end{tabular}
\end{table}

需求表反映了不同价格下消费者对$x$商品的不同需求量。根据\eqref{eq:2-4}式或表2-1可以画出消费者的需求曲线,如下图:

需求函数、需求表或需求曲线都是在某特定时期内的需求量与价格之间的关系,通常的需求曲线,不管是线性的需求曲线还是非线性的需求曲线,都是向右下方倾斜的。需求曲线向右下方倾斜的状况揭示了需求函数的一个重要持征,即需求量与价格呈反方向变化,这一特征被称为{\hei 需求法则}(law of demand)。需求法则表述如下:假定其他条件不变(ceteris paribus),商品的需求量与其价格呈反方向变化;价格上升,需求量下降;价格下降,需求量上升。用数学语言讲,需求法则表明需求量对于价格的一阶导数小于零,即:$dq_x^d/dP_x<0$

需求函数、需求表或需求曲线所刻画的是商品的需求量与商品的价格之间的关系,表明在各种不同价格下的需求量,一旦商品的价格发生变化,商品的需求量将随之变化。这种因商品自身的价格变化而引起的商品需求量的变化称为需求量的变动,在需求理论中,研究需求量的变动是很重要的。在局部均衡的条件下,需求理论中的另一种变动,即需求的变动也是值得探讨的,需求的变动是由于除了商品自身的价格以外其他因素的变化引起的。例如,消费者收入的变动、消费者偏好的变动、其他商品价格的变动等因素都会引起需求的变动。要区分需求的变动与需求量的变动,需求量的变动由于是商品自身价格的变化引起的,因此它并不改变需求函数或需求表,而由于需求的变动是除了商品自身价格以外的其他因素的变化所引起的,因此它的变动是整个需求函数或整个需求表的变动。从需求曲线来看,需求量的变动是同一条需求曲线上不同点的变动;需求的变动则是整条需求曲线的移动。在图2-2中,从$a$点到$b$点的变动是需求量的变动;而从曲线$D_1$到$D_2$的变动则是需求的变动。

对于影响需求的变动的因素可以进行具体分析。例如,消费者收入的变动如何影响需求的变动视商品是{\hei 正常品}(normal goods)与{\hei 劣等品}(inferior goods)而定。如果消费者所购买的商品是{\hei 正常品},那么需求的变动与消费者的收入同方向变化,即消费者收入的增加将引起需求的提高,消费者收入的减少将引起需求的降低。如果消费者所购买的商品是{\hei 劣等品},那么需求的变动与消费者的收入反方向变化,即消费者收入的增加引起需求的降低,消费者收入的减少引起需求的提高。日常生活中我们可以看到,消费者对于精细食品或高中档服装的需求与其收入同方向变化;而消费者对于粗粮与粗布等商品的需求与消费者的收入则呈反方向变化。

其他商品价格的变化对消费者所购买的商品需求的影响分商品是{\hei 替代品}(substitutes)还是{\hei 补充品}(complements)而有所不同。{\hei 替代品}是指消费中可以相互替代以满足消费者某种欲望的商品,{\hei 补充品}是指在消费中可以相互补充以满足消费者某种欲望的商品。例如,苹果与梨是替代品,咖啡与茶是替代品;汽油是汽车的补充品,糖是咖啡的补充品。如果$A$商品与$B$商品是替代品,那么$A$商品的需求与$B$商品价格同方向变化,即商品价格的提高将引起$A$商品需求的增加,$B$商品价格的降低将引起$A$商品需求的减少。如果$B$商品是$A$商品的补充品,那么$A$商品的需求将与$B$商品的价格反方向变化,即$B$商品价格的提高将引起$A$商品需求的降低,$B$商品价格的降低将引起商品需求的提高。

\subsection{市场需求}

市场需求定义如下:{\hei 市场需求}是一个表列,它表示在某一特定市场和某一特定时期内,所有购买者在各种可能的价格下将购买的某种商品的各种数量。

市场需求是个人需求的加总。令$Q_x^d$表示x商品的市场需求,$q_x^{di}$表示第$i$个消费者对于x商品的需求。假定市场上有$n$个消费者,则

\begin{equation}\label{eq:2-5}
	Q_x^d = \sum_{i=1}^nq_x^{d_i}	
\end{equation}

假定所有消费者对于$x$商品的函数都是相同的,那么包括了$n$个消费者的市场需求函数就是单个消费者的需求函数乘以$n$以\eqref{eq:2-4}式的需求函数$q_x^d=8-p_x$为例,如果该商品市场上有10000个消费者,每个消费者的需求函数相同。对于该商品的市场需求为:

\begin{equation}\label{eq:2-6}
	Q_x^d = 80000 - 10000p_x
\end{equation}


对应该需求函数的市场需求表为:

\begin{tabular}{|c|c|c|c|c|c|c|c|c|}
	\hline
	$p_x$ 	& 0 & 1 & 2 & 3 & 4 & 5 & 6 & 7 \\
	\hline
	$Q_x^d$	& 80000 & 70000 & 60000 & 50000 & 40000 & 30000 & 20000 & 10000 \\
	\hline
\end{tabular}

与表2-2相对应的市场需求曲线为:

\setlength{\unitlength}{1cm}
\begin{picture}(5,5)
	\put(0,0){\line(0,1){5}}
	\put(0,0){\line(1,0){5}}
	\put(0,4){A}
	\put(4,0){B}
	\put(0,4){\line(1,-1){4}}
	\put(0,4){\line(1,-2){2}}
\end{picture}

图2-3的形状与图2-2相同,纵坐标完全一样,但图2-3的横坐标所表示的比例单位比图2-2横坐标所表示的单位要大得多。

由于市场需求是个人需求的加总,因此,凡是影响个人需求的因素都会影响市场需求。此外,市场需求还受消费者人数多寡的影响。有些情况下,某种商品的价格降低后,每个消费者对于该商品的需求量都增加了,从而市场需求量也增加了;但另一些情况下,某种商品的价格降低后,市场需求量的增加并不是由于原有的消费者消费数量增加了,而是由于消费该商品的消费者数目增加了。小说、报刊、杂志等类产品的消费就存在这类情况。原来只订阅一份某种报刊的某消费者在该报刊降低后不会订阅两份同样的报刊。而原来不订阅该报刊的消费者在该报刊降价后则有可能订阅这种报刊。因此,该报刊市场需求量的增加是消费者数目增加的结果。

与个人需求曲线的形状相同,市场需求曲线也是向右下方倾斜的。因此,对于市场需求来说,需求法则也成立。但是,无论是个人需求还是市场需求都存在例外的情况,即需求曲线不是向右下方倾斜的情况。例如{\hei 吉芬商品} (Giffen goods)的需求曲线是向右上方倾斜的。吉芬商品是以19世纪英国的经济学家吉芬(Robert Giffen)的名字命名的。吉芬发现,当时土豆的价格上升,但是土豆的需求量却在增加。土豆的需求量与价格同方向变化。吉芬商品是唯一不遵从需求法则的特殊商品。关于吉芬商品违背需求法则的原因,我们将在与消费者行为理论有关的收入效应与替代效应的讨论中详细分析。


\section{供给}

像上一节分别讨论个人需求与市场需求一样,本节我们将分别讨论单个厂商的供给与市场供给。

\subsection{单个厂商的供给}

{\hei 单个厂商的供给}是指在某一特定时期内,单个生产者在各种可能的价格下愿意并且能够出售的某种商品的各种数量。

单个厂商的供给受多种因素的影响,其中主要有:厂商打算出售的商品的价格,为生产该商品厂商所投入的生产要素的成本,厂商的技术状况,厂商对于商品未来价格的预期,其他商品的价格等等。我们可以在厂商的供给与影响厂商供给的因素之间建立一种函数关系,称为{\hei 单个厂商的供给函数}。就上述几种影响厂商供给的因素而言,单个厂商的供给函数一般形式是:

\begin{equation} \label{eq:2-7}
	q_x^s = f(P_x; P_1,P_2,\dots,P_n; P_e; c; \rho)
\end{equation}



\newpage
