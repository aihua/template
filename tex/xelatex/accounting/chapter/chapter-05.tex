\chapter{差分与差分方程}

	\section{差分与差分方程的基本概念}

		\subsection{差分的概念}

		自然科学与工程技术中遇到的常是连续变量的函数,它的变化率用导数来描述。而经济中遇到的变量常是离散的,例如以年度或月,或者以某个时间段作为一期来考虑,讨论某个经济量按期的变化规律。这就引出以整数集(或者非负整数集)作为定义域的函数以及用上期与这期的差来描述其变化规律。

		\definition 设函数$y(t)$的{\heiti 定义域为非负整数}$N$,常将$y(t)$在$t$处的值记为$y_t$,或者干脆用$y_t$来表示$y(t)$。函数$y_t$从$y_t$到$y_{t+1}$的增量$y_{t+1} - y_t $称为$y_t$在$t$处的{\heiti 一阶差分},记为:$$\Delta y_t = y_{t+1} - y_t$$

		简称为$y_t$在$t$处的{\heiti 差分}。$y_t$在$t$处的一阶差分的差分,称为$y_t$在$t$处的{\heiti 二阶差分},记为$\Delta ^ 2 y_t$,有$$\Delta ^ 2 y_t = \Delta(\Delta y_t) = \Delta(y_{t+1} - y_t) = y_{t+2} - y_{t+1} - (y_{t+1} - y_t) = y_{t+2} - 2y_{t+1} + y_t$$

		用归纳法可以定义$y_t$在$t$处的$n$阶差分:$$\Delta ^n y_t = \Delta(\Delta ^ {n-1} y_t), \quad n=2,3,\ldots $$

		容易证明

		\begin{equation}\label{eq:5-1}
			\Delta ^ n y_t = \sum^n_{i=0} (-1)^i \binom{n}{i} y_{t+n-i}, \quad n = 1,2,\ldots
		\end{equation}

		当$n \geqslant 2$时,$\Delta ^ n y_t$通称为$y_t$的{\heiti 高阶差分}。

		\subsection{差分的四则运算公式与某些基本初等函数的差分}

		由一阶差分的定义,容易证明差分的四则运算法则:

		\begin{enumerate}
			\item $\Delta(cy_t) = c \Delta y_t$ ($c$为常数);
			\item $\Delta(y_t \pm z_t) = \Delta y_t \pm \Delta z_t$;
			\item $\Delta(y_t\cdot z_t) = y_{t+1} \Delta z_t + z_t \Delta y_t = z_{t+1} \Delta y_t + y_t \Delta z_t $;
			\item $\Delta \left( \dfrac{y_t}{z_t} \right) = \dfrac{z_t \Delta y_t - y_t \Delta z_t}{z_tz_{t+1}} = \dfrac{z_{t+1} \Delta y_t - y_{t+1} \Delta z_t}{z_t z_{t+1}}$ (当$z_t z_{t+1} \neq 0$)。
		\end{enumerate}

		其中(1)与(2)是显然的,今证(3),(4)是类似的。

		(3)的证明:$\Delta(y_t \cdot z_t) = y_{t+1}z_{t+1} - y_tz_t = y_{t+1}z_{t+1}- y_{t+1}z_t + y_{t+1}z_t - y_tz_t = y_{t+1}(z_{t+1} - z_t) + z_t(y_{t+1} - y_t) = y_{t+1} \Delta z_t + z_t \Delta y_t$

		类似地可证第二个等式。证毕。

		\example\label{example:1-1} 设$y_t \equiv c$($c$为常数),则$\Delta y_t = y_{t+1} - y_t = c - c = 0$,即有$\Delta c = 0$,常数的差分为$0$。

		\example 设$y_t = t^n$,$n$为正数,求$\Delta y_t$及$\Delta ^{n+1} y_t$。

		\answer $\Delta y_t = (t + 1)^n - t^n = \sum ^n_{i=0} \binom{n}{i} t^{n - i} - t^n = \sum^n_{i=1} \binom{n}{i} t^{n - i}$。

		右边最高次项为$t^{n - 1}$,所以$\Delta t^n$为$t$的一个$n - 1$次多项式。

		再由四则运算法则(1)与(2)可知,$\Delta ^2 t^n$为$t$的一个$n - 2$次多项式,于是推知,$\Delta ^{n + 1} t^n$为$0$。

		\example 设$y_t = a^t$,$a$为常数(若$a = 0$,则$t$为正整数),求$\Delta y_t$及$\Delta ^m y_t$($m$为某正整数)。

		\answer $\Delta y_t = y_{t + 1} - y_t = a^{t + 1} - a^t = a^t(a - 1)$。
		
		(此式当$a = 1$时亦成立。因为此时,$y_t = 1$,由例\ref{example:1-1}有$\Delta y_t = 0$。再由归纳法有$$\Delta ^m y_t = a^t(a - 1)^m, \quad (m = 1,2,\ldots)$$

		\example 设$\Delta y_t = \sin at $,$a$为常数且$a \neq 0$,求$\Delta y_t$。

		\answer $\Delta y_t = \sin a (t + 1) - \sin at  = 2 \sin \dfrac{a}{2}  \cos a (t + \dfrac{1}{2})$

	\subsection{差分方程的基本概念}

		\definition 设$y_t$是未知函数,方程
		\begin{equation}\label{eq:5-2}
			H(t, y_t, \Delta y_t, \Delta ^2 y_t, \ldots, \Delta ^n y_t) = 0
		\end{equation}
		或
		\begin{equation}\label{eq:5-3}
			F(t, y_t, y_{t + 1}, \ldots, y_{t + n} = 0
		\end{equation}
		称为以$t$为自变量以$y_t$为未知函数的{\heiti 差分方程},其中$H$与$F$是变元的已知函数,$n \geqslant 1$。

		由(\ref{eq:5-2})式可知,$\Delta ^ k y_t$可以由$y_t,y_{t + 1}, \ldots, y_{t + k}$线性表出,故(\ref{eq:5-2})式可化为(\ref{eq:5-3})式。以后我们只讨论(\ref{eq:5-3})式。

		设(\ref{eq:5-3})中一定含有下标最大的$y_{t + n}$与下标最小的$y_t$。最大下标$(t + n)$与最小下标$t$的差为$n$,$n \geqslant 1$,则称(\ref{eq:5-3})使成为恒等式:$$ F(t, \varphi(t), \varphi(t + 1), \ldots, \varphi(t + n)) \equiv 0 $$,则称$y_t = \varphi (t)$为(\ref{eq:5-3})的{\heiti 解}。

		\definition 类似于微分方程,有通解、特解等定义,$n$阶差分方程的初始条件是这样提的:设$t_0$为某非负整数,下面这组条件
		\begin{equation}
			y(t_0) = y_0, \quad y(t_0 + 1) = y_1, \quad\ldots, \quad y (t_0 + n -1) = y_{n - 1}
		\end{equation}
		称为$n$阶差分方程的{\heiti 初始条件}。

		\example\label{example:1-5} 设$c$是常数,验证$y_t = c2^t - t - 1$为一阶差分方程$y_{t + 1} - 2y_t = t$的解。

		\answer 由$y_t = c2^t - t- 1$,有$y_{t + 1} = c2^{t + 1} - (t + 1) -1$,代入所给差分方程得$$c2^{t + 1} - (t + 1) - 2(c2^t - t - 1) = t$$故知$y_t = c2^t - t - 1$的确是$y_{t + 1} - 2y_t = t$的解。

		\example\label{example:1-6} 验证$y_t = \dfrac{1}{c + t}$是差分方程$(1 + y_t)y_{t + 1} = y_t$的解,其中$c$是常数。

		\answer 由$y_t = \dfrac{1}{c + t }$有$y_{t + 1} = \dfrac{1}{c + 1 + t}$,于是$$(1 + y_t) = \left(1 + \frac{1}{c + t} \right) \frac{1}{c + t +1} = \frac{1}{c + t} = y_t$$验毕。

		\remark 例\ref{example:1-5}与例\ref{example:1-6}中,若$c$为任意常数,则$y_t = c2^t - t - 1$与$y_t = \dfrac{1}{c + t}$分别是例\ref{example:1-5}与例\ref{example:1-6}这两个方程的通解。在例\ref{example:1-5}中如果给了初始条件$y_t \vert _{t=0} = y_0$,则由$y_0 = c2^0 - 0 - 1$,有$c = y_0 + 1$,从而得例\ref{example:1-5}的特解$y_t = (y_0 + 1) 2^t -t - 1$。又如在例\ref{example:1-6}中如果给了初始条件$y  \vert _{t=0} = y_0 \neq 0$,则由$y_0 = \dfrac{1}{c + 0}$有$c = \dfrac{1}{y_0}$,从而得到特解$y_t = \dfrac{y_0}{1 + y_0t}$(当$y_0 \neq 0$)。但若$y_0 = 0$,易见有特解$y_t \equiv 0$,但从通解$y_t = \dfrac{1}{c + t}$中得不出这个特解,可见通解并非一切解。这种情况在微分方程中也见到过。

	\section{一阶及二阶常系数线性差分方程的解法}

		本章只讨论一阶及二阶常数系数线性差分方程的解法。由于$n$阶线性差分方程的解的结构十分简单,且与$n$阶线性常微分方程的解的结构类似,故先作一点关于它的介绍。

		\subsection{$n$阶线性差分方程及其解的结构}

		\definition 方程
		\begin{equation}\label{eq:5-4}
			y_{t+n} + a_1(t)y_{t+n-1} + \ldots + a_{n-1}(t)y_{t+1} + a_n(t)y_t = f(t)
		\end{equation}
		称为$n$阶线性{\heiti 差分方程},其中$t \in N$,$a_1(t), a_2(t), \ldots, a_n(t)$与$f(t)$均为$t$的已知函数,且$a_n(t) \neq 0$。

		关于自由项,非齐次方程,齐次方程
		\begin{equation}\label{eq:5-5}
			y_{t+n} + a_1(t)y_{t+n-1} + \ldots + a_{n-1}(t)y_{t+1} + a_n(t)y_t = 0
		\end{equation}
		与非齐次方程对应的齐次方程、变系数、常系数等名称,以及线性相关、线性无关等概念均与常微分方程中所讲的相同,在此均略。

		与常微分方程类似,有

		\theorem\label{theorem:5-1} 设函数$y_1(t), y_2(t), \ldots, y_m(t) (t \in N)$是(\ref{eq:5-5})的$m$个解,$c_1, c_2, \ldots, c_m$是$m$个常数,则$$y_t = \sum^{m}_{i=1}c_iy_i(t)$$也是(\ref{eq:5-5})的解。

		\theorem\label{theorem:5-2} 设函数$y_1(t), y_2(t), \ldots, y_n(t) (t \in N)$是(\ref{eq:5-5})的$n$个线性无关的解,\\
		$c_1, c_2, \ldots, c_n$是$n$个任意常数,则
		\begin{equation}\label{eq:5-6}
			y_t = \sum ^n _{t=1} c_i y_i (t)
		\end{equation}
		是(\ref{eq:5-5})的通解。

		\theorem\label{theorem:5-3} 设$y^*_t$是(\ref{eq:5-4})的一个特解,$Y_t$是(\ref{eq:5-4})对应的(\ref{eq:5-5})的通解,则$$y_t = Y_t + y^*_t$$是(\ref{eq:5-4})的通解。

		\theorem\label{theorem:5-4} 设$y^*_1(t)$与$y^*_2(t)$分别是$$y_{t+n} + a_1(t)y_{t+n-1} + \ldots + a_{n-1}(t)y_{t+1} + a_n(t)y_t = f_1(t)$$与$$y_{t+n} + a_1(t)y_{t+n-1} + \ldots + a_{n-1}(t)y_{t+1} + a_n(t)y_t = f_2(t)$$的解,则$$y_t = y^*_1(t) + y^*_2(t)$$是$$y_{t+n} + a_1(t)y_{t+n-1} + \ldots + a_{n-1}(t)y_{t+1} + a_n(t)y_t = f_1(t) + f_2(t)$$的解。

		\remark 线性方程的通解与一切解是一致的。

		下面讨论线性常系数一阶、二阶差分方程的解法。

		\subsection{一阶常系数线性差分方程的解法}

		\definition 设$p$是常数,$p \neq 0$,$f(t)$是$t$的已知函数,$t \in N$。方程
		\begin{equation}\label{eq:5-7}
			y_{t+1} + py_t = f(t)
		\end{equation}
		与
		\begin{equation}\label{eq:5-8}
			y_{t+1} + py_t = 0
		\end{equation}
		分别称为{\heiti 一阶常系数线性非齐次差分方程}与{\heiti 齐次差分方程}。

		先讨论(\ref{eq:5-8}的解法。

		设想一个函数$y_t$,将它代入(\ref{eq:5-8}),它与$y_{t+1}$会相消。试以$y_t = \lambda ^t $代入计算之,得$$ \lambda ^{t+1} + p \lambda ^t = 0$$,约去$\lambda ^t$,得
		\begin{equation}\label{eq:5-9}
			\lambda + p = 0
		\end{equation}
		可见,当$\lambda = -p$时,$y_t = (-p) ^t$满足(\ref{eq:5-8})式,从而知$y_t = (-p) ^t$是(\ref{eq:5-8})的一个非零解。由定理\ref{theorem:5-2}知,$c(-p)^t$为(\ref{eq:5-8})的通解,其中$c$是任意常数。

		方程(\ref{eq:5-9})称为(\ref{eq:5-8})的特征方程。特征方程的根$\lambda = -p$称为(\ref{eq:5-8})的特征根,由特征根便可写出(\ref{eq:5-8})的通解$y_t = c(-p)^t$。这也就是解(\ref{eq:5-8})的步骤。

		\example 求差分方程$2y_{t+1} - 3y_t = 0$的通解及满足初始条件$y(0) = y_0$的特解。

		\answer 写成(\ref{eq:5-8})的形式,为$$y_{t+1} - \frac{3}{2}y_t = 0$$,特征方程为$$\lambda - \frac{3}{2} = 0$$,特征根$\lambda = \dfrac{3}{2}$,所以通解为$y_t = c\left(\dfrac{3}{2}\right) ^t$。

		再由初始条件$y_0 = c\left(\dfrac{3}{2}\right) ^0 = c$,所以特解为$y_t = y_0\left(\dfrac{3}{2}\right) ^t$。

		下面用待定系数法讨论一些特殊自由项$f(t)$的非齐次方程(\ref{eq:5-7})的解法。

		{\heiti 类型1} \quad 设$f(t)$为$t$的已知的$m$次多项式$P_m(t)$情形,即讨论求
		\begin{equation}\label{eq:5-10}
			y_{t+1} + py_t = P_m(t)
		\end{equation}
		的解。由$y_{t+1} = y_t + \Delta y_t$,代入上式,(\ref{eq:5-10})$$\Delta y_t + (p + 1) y_t = P_m(t)$$。因右边$P_m(t)$为$t$的$m$次多项式,故可设想$y_t$是一个多项式。因$\Delta y_t$为比$y_t$低一次的多项式,于是推知,若$-p \neq 1$,则设$y_t$为一个$m$次多项式$Q_m(t)$,系数待定;若$-p = 1$,则应设$y_t$为一个$m + 1$次多项式。因常数的差分为$0$,故此时可设$y_t = tQ_m(t)$,$Q_m(t)$的意义同上。总结以上两种情况,可设特解为
		\begin{equation}\label{eq:5-11}
			y^*_t = t^kQ_m(t)
		\end{equation}
		其中,$Q_m(t)$为$t$的$m$次多项式,系数待定。
		\begin{displaymath}
			k = 
			\begin{cases}
				0 & \text{当$-p \neq 1$(即$1$不是特征根)时;} \\
				1 & \text{当$-p = 1$(即$1$是特征根)时。}
			\end{cases}
		\end{displaymath}

		\example 求方程$2y_{t + 1} + y_t = t^2$的通解。

		\answer 改写为$y_{t + 1} + \dfrac{1}{2} y_t = \dfrac{1}{2} t^2$,故知特征根$\lambda = -\dfrac{1}{2}$,对应齐次方程的通解为$Y_t = c\left(-\dfrac{1}{2}\right)^t$。自由项为$t$的$2$次多项式,特征根$\neq 1$,故可令一个特解为$$y^*_t = At^2 + Bt + C$$。代入所给方程,得$$2\left(A(t + 1)^2 + B(t + 1) + C\right) + (At^2 + Bt + C) = t^2$$。经整理,得$$3At^2 + (4A + 3B)t + (2A + 2B + 3C) = t^2$$。比较系数解得$A=\dfrac{1}{3}$,$B=-\dfrac{4}{9}$,$C=\dfrac{2}{27}$,从而得$$y^*_t = \dfrac{1}{3}t^2 - \dfrac{4}{9}t + \dfrac{2}{27}$$,于是通解为$y_t = c \left(-\dfrac{1}{2}\right)^t + \dfrac{1}{3}t^2 - \dfrac{4}{9}t + \dfrac{2}{27}$。

		\example 求方程$y_{t+1} - y_t =t$的通解。

		\answer 特征方程$\lambda - 1 = 0$,特征根$\lambda = 1$,故对应的齐次方程的通解为$Y_t = c \cdot 1^t = c$。自由项为$t$的一次多项式,$1$是特征根,故可令一个特解为$$y^*_t = t(At + B) = At^2 + Bt$$,代入原方程,得$$A(t+1)^2 + B(t+1) - (At^2 + Bt) = t$$,整理得$$2At + A + B = t$$,比较系数得$A=\dfrac{1}{2}$,$B=-\dfrac{1}{2}$。从而得$y^*_t = \dfrac{1}{2}t^2 - \dfrac{1}{2}t$,于是通解为$y_t = Y_t + y^*_t = c + \dfrac{1}{2} t^2 - \dfrac{1}{2} t$,其中$c$为任意常数。

		{\heiti 类型2}\label{prop:5-2} 设$f(t)$为$P_m(t)a^t$的情形,其中$P_m(t)$为$t$的已知的$m$次多项式,常数$a \neq 0$,$a \neq 1$。即讨论求
		\begin{equation}\label{eq:5-12}
			y_{t+1} + py_t = P_m(t)a^t
		\end{equation}
		的解。此时可设特解为$$y^*_t = t^kQ_m(t)a^t$$,其中$Q_m(t)$为$t$的$m$次多项式,系数待定,
		\begin{displaymath}
			k = 
			\begin{cases}
				0 & \text{当$-p \neq a$(即$a$不是特征根)时;} \\
				1 & \text{当$-p = a$(即$a$是特征根)时。}
			\end{cases}
		\end{displaymath}

		\begin{proof}
		对(\ref{eq:5-12})作未知函数的变量变换,令
		\begin{equation}\label{eq:5-13}
			y_t = a^tz_t \text{,}
		\end{equation}
		于是(\ref{eq:5-12})成为$$a^{t+1}z_{t+1} + pa^tz_t = P_m(t)a^t$$,消去$a^t$,得
		\begin{equation}\label{eq:5-14}
			z_{t+1} + \dfrac{p}{a} z_t = \dfrac{1}{a}p_m(t)
		\end{equation}
		此为{\heiti 类型2}的情形。对照{\heiti 类型2}的结论,当$-\dfrac{p}{a} \neq 1$时,令$z^*_t = Q_m(t)$,即令$y^*_t = Q_m(t)a^t$;当$-\dfrac{p}{a}=1$时,令$z^*_t = tQ_m(t)$,即令$y^*_t = tQ_m(t)a^t$。
		\end{proof}

		\example 求差分方程$y_{t+1} + y_t = te^t$的通解。

		\answer 特征方程为$\lambda + 1 = 0$,特征根$\lambda = -1$,对应齐次方程的通解为$Y_t = c(-1)^t$。$a = e \neq -1$(特征根为$-1$),令特解$y^*_t = (At + B)e^t$,代入原方程,得$$\left( A(t+1) + B \right) e^{t+1} + (At + B)e^t = te^t$$,消去$e^t$并整理,得$$A(e+1)t + eA + eB + B = t$$,得$A = \dfrac{1}{e+1}$,$B=-\dfrac{e}{(e+1)^2}$,从而得$y^*_t = \left(\dfrac{t}{e+1} - \dfrac{e}{(e+1)^2}\right)e^t$,通解$y_t = Y_t + y^*_t = c(-1)^t + \left(\dfrac{t}{e+1} - \dfrac{e}{(e+1)^2}\right) e^t$,其中$c$为任意常数。

		\example 求差分方程$y_{t+1} - 2y_t = 2^t$的通解。

		\answer 特征方程为$\lambda - 2 = 0$,特征根$\lambda = 2$,对应齐次方程的通解为$Y_t = c \cdot 2^t$,$a = 2 \text{特征根} 2$,令特解$y^*_t = t(A \cdot 2^t) = A \cdot t \cdot 2^t$,代入原方程,得$$A(t+1)2^{t+1} - 2A \cdot t \cdot 2^t = 2^t$$,约去$2^t$,并整理,得$2A=1$,$A=\dfrac{1}{2}$。从而得$y^*_t = \dfrac{1}{2} \cdot t \cdot 2^t$,故得通解$y_t = Y_t + y^*_t = c \cdot 2^t + t \cdot 2^{t-1}$。

		\example 设$\alpha$,$\beta$均为常数,$\alpha \neq 0$,试讨论差分方程$y_{t+1} - \alpha y_t = e^{\beta t}$的通解。

		\answer 特征方程$\lambda - \alpha = 0$,特征根$\lambda = \alpha$,对应齐次方程的通解$Y_t = c \alpha ^t$。为讨论非齐次方程的通解,改写原方程为$y_{t+1} - \alpha y_t = (e^{\beta})^t$,

		(1)如果$e^{\beta} = \alpha$,则令$y_{t+1} = A(e^{\beta})^t$,代入原方程,得$A(e^{\beta})^{t+1} - \alpha A(e^{\beta})^t = (e^{\alpha}) ^ t$,约去$e^{\beta t}$,得$$A(e^{\beta} - \alpha) = 1 \text{,} A = \dfrac{1}{e^{\beta} - \alpha}$$,$$y^*_t = \dfrac{1}{e^{\beta} - \alpha}e^{\beta t}$$,通解$y_t = Y_t + y^*_t = c {\alpha}^t + \dfrac{1}{e ^ {\beta} - \alpha} e^{\beta t}$。

		(2)如果$e^{\beta} = \alpha$,则令$y^*_t = At(e^{\beta})^t$,代入原方程,得$$A(t+1)e^{\beta (t+1)} - \alpha Ate^{\beta t} = e^{\beta t}$$,约去$e^{\beta t}$,并注意到$\alpha = e^{\beta}$,得$Ae^{\beta} = 1$,$A = e^{-\beta}$。得$y^*_t = t e^{\beta (t-1)}$,通解$y_t = Y_t + y^*_t = (c + te^{-\beta})e^{\beta t}$。

		\subsection{二阶常系数线性差分方程的解法}

		\definition 设$p$与$q$都是常数,$q \neq 0$,$f(t)$是$t$的已知函数,$t \in N$,方程
		\begin{equation}\label{eq:5-15}
			y_{t+2} + py_{t+1} + qy_t = f(t)
		\end{equation}
		与
		\begin{equation}\label{eq:5-16}
			y_{t+2} + py_{t+1} + qy_t = 0
		\end{equation}
		分别称为{\heiti 二阶常系数线性非齐次差分方程}与{\heiti 齐次差分方程}。

		先讨论(\ref{eq:5-16})的解法。

		仿一阶常系数线性差分方程,试以$y_t = \lambda ^t$代入(\ref{eq:5-15})式,并约去$\lambda ^t$,得
		\begin{equation}\label{eq:5-17}
			\lambda ^2 + p\lambda + q = 0
		\end{equation}
		(\ref{eq:5-17})称{\heiti 特征方程},它的根$\lambda = \lambda _1$与$\lambda _2$称为{\heiti 特征根}。

		(1)若$\lambda _1$与$\lambda _2$为不相等的实根,则易知$\lambda ^t_1$与$\lambda ^t_2$均是(\ref{eq:5-15})的解,且$\lambda ^t_1$与$\lambda ^t_2$线性无关,由定理\ref{theorem:5-2}知,$c_1 \lambda ^t_1 + c_2 \lambda ^t_2$为(\ref{eq:5-15})的通解,其中$c_1$与$c_2$为两个任意常数。

		(2)若$\lambda _1 = \lambda _2$为二重特征根,容易证明,此时对应了两个线性无关的特解$\lambda ^t_1$与$t\lambda ^t_1$,由定理\ref{theorem:5-2}知,$(c_1 + c_2t) \lambda ^t_1$为(\ref{eq:5-16})的通解,其中$c_1$与$c_2$为任意常数。

		(3)若$\lambda _{1,2} = \alpha \pm i\beta(\beta > 0)$为一对共轭复数根,$(\alpha + i\beta)^t$与$(\alpha - i\beta)^t$分别是(\ref{eq:5-15})的解。但复数形式使用不便,改换成三角式表示:
		\begin{equation}\label{eq:5-18}
			\alpha \pm i\beta = r \cos \theta \pm ir \sin \theta
		\end{equation}
		其中,
		\begin{equation}\label{eq:5-19}
			r = \sqrt{\alpha ^ 2 + \beta ^ 2}, \quad \cos \theta = \dfrac{\alpha}{r}, \quad \sin \theta = \dfrac{\beta}{r}
		\end{equation}
		再由棣莫弗(de Moivre)公式,$$(\alpha \pm i\beta) ^ t = (r \cos \theta \pm ir \sin \theta) ^t = r^t \cos t \theta \pm ir^t \sin t\theta$$。容易证明,在这种情形下,$r^t \cos t \theta$与$r^t \sin t \theta$分别是(\ref{eq:5-15})的两个线性无关的解,$$(c_1 \cos t\theta + c_2 \sin t\theta)r^t$$是(\ref{eq:5-15})的通解,其中$c_1$与$c_2$为两个任意常数。

		将上述结论,列表于后。

		\example 求$y_{t+2} + 2y_{t+1} - 3y_t = 0$的通解。

		\answer 特征方程$\lambda ^2 + 2\lambda - 3 = 0$,特征根$\lambda _1 = 1$,$\lambda _2 = -3$,通解为$y_t = c_1 + c_2(-3)^t$。

		\example 求$y_{t+2} + 2y_{t+1} + y_t = 0$的通解。

		\answer 特征方程$\lambda ^2 + 2\lambda + 1 =0$,特征根$\lambda _{1,2} = -1$(二重根),通解为$y_t = (c_1 + c_2 t)(-1)^t$。

		\example 求$y_{t+2} + y_{t+1} + y_t = 0$的通解。

		\answer 特征方程$\lambda ^2 + \lambda + 1 = 0$,特征根$\lambda _{1,2} = -\dfrac{1}{2} \pm i \dfrac{\sqrt{3}}{2}$为一对共轭复数根,由(\ref{eq:5-18})$$r=1, \quad \cos \theta = -\frac{1}{2}, \quad \sin \theta = \frac{\sqrt{3}}{2}, \quad \theta = \frac{2 \pi}{3}$$,通解为$y_t = c_1 \cos \dfrac{2\pi}{3}t + c_2 \sin \dfrac{2\pi}{3}t$。

		下面用待定系数发讨论一些特殊自由项$f(t)$的非齐次方程(\ref{eq:5-14})的解法。为节省篇幅,以下只介绍方法,推理从略。

		{\heiti 类型1} \quad 设$f(t)$是$t$的已知的$m$次多项式$P_m(t)$情形,即讨论求$$y_{t+2} + py_{t+1} + qy_t = P_m(t)$$的解。此时,可设特解为$$y^*_t = t^k Q_m(t)$$,其中$Q_m(t)$为$t$的$m$次多项式,系数待定。
		\begin{displaymath}
			k = 
			\begin{cases}
				0 & \text{当$1 + p + q \neq 0$(即$1$不是特征根)时;} \\
				1 & \text{当$1 + p + 1 = 0$,而$2 + p \neq 0$(即$1$是单重特征根)时;} \\
				2 & \text{当$1 + p + 1 = 0$,$2 + p = 0$(即$1$是二重特征根)时。}
			\end{cases}
		\end{displaymath}
		\example 求$y_{t+2} + 2y_{t+1} - 3y_t = t$的通解。

		\answer 特征方程$\lambda ^2 + 2\lambda - 3 = (\lambda - 1)(\lambda + 3) = 0$,特征根$\lambda _1 = 1$,$\lambda _2 = -3$,故对应齐次方程的通解为$Y_t = c_1 + c_2(-3)^t$。又因$1$是单重特征根,故可令特解$y^*_t = t(At + B)$,代入原方程得$(t+2)(A(t+2)+B) + 2(t+1)(A(t+1) + B) -3t(At + B) = t$,整理之,得$$8At + 6A + 4B = t$$,故$A=\dfrac{1}{8}$,$B=-\dfrac{3}{16}$,$y^*_t = t(\dfrac{1}{8} t - \dfrac{3}{16})$,故原方程的通解为$y_t = Y_t + y^*_t = c_1 + c_2(-3)^t + t(\dfrac{1}{8} t - \dfrac{3}{16})$。

		{\heiti 类型2} \quad 设$f(t)$为$P_m(t)a^t$的情形,其中$P_m(t)$为$t$的已知$m$次多项式,常数$a \neq 0$,$a \neq 1$,即讨论求
		\begin{equation}\label{eq:5-20}
			y_{t+2} + py_{t+1} + qy_t = P_m(t)a^t
		\end{equation}
		的解。此时,可设特解为$$y^*_t = t^kQ_m(t)a^t$$,其中$Q_m(t)$为$t$的$m$次多项式,系数待定。
		\begin{displaymath}
			k = 
			\begin{cases}
				0 & \text{当$a^2 + pa + q \neq 0$(即$a$不是特征根)时;} \\
				1 & \text{当$a^2 + pa + q = 0$,而$2a + p \neq 0$(即$a$是单重特征根)时;} \\
				2 & \text{当$a^2 + pa + q = 0$,$2a + p = 0$(即$a$是二重特征根)时。}
			\end{cases}
		\end{displaymath}
		\example 求$y_{t+2} - 2y_{t+1} - 3y_t = 3^t$的通解。

		\answer 特征方程$\lambda ^2 - 2\lambda -3 = (\lambda -3)(\lambda + 1) = 0$,特征根$\lambda _1 = 3$,$\lambda _2 = -1$,对应齐次方程的通解为$Y_t = c_1 3^t + c_2 (-1)^t$,由于$c=3$为单重特征根,故可令一个特解为$$y^*_t = tA3^t = At3^t$$,代入原方程并约去$3^t$,得$$9A(t+2) - 6A(t+1) -3At = 1$$,解得$A = \dfrac{1}{12}$,故通解为$y_t = Y_t + y^*_t = c_1 3^t + c_2 (-1)^t + \dfrac{1}{12}t 3^t$。

		\example 求$y_{t+2} - 3y_{t+1} + 2y_t = 2 + t 3^t$的通解。

		\answer 特征方程$\lambda ^ 2 - 3 \lambda + 2 = 0$,特征根$\lambda _1 = 1$,$\lambda _2 = 2$,对应齐次方程的通解为$Y_t = c_1 + c_2 2^t$。为求原方程的一个特解,应分别考虑如下两个方程:
		\begin{equation}\label{eq:5-21}
			y_{t+2} - 3y_{t+1} + 2y_t = 2
		\end{equation}
		与
		\begin{equation}\label{eq:5-22}
			y_{t+2} - 3y_{t+1} + 2y_t = t 3^t
		\end{equation}

		对于第1个方程,属于{\heiti 类型1},“$1$”是单重特征根,故应令$$y^{*(1)}_t = At$$,代入(\ref{eq:5-21})得$$A(t+2) - 3A(t+1) + 2At = 2$$,得$A = -1$,对应的$y^{*(1)}_t = -t$。

		
		对于第2个方程,属于{\heiti 类型2},$a=3$不是特征根,故应令$$y^{*(2)}_t = (Bt + C) 3^t$$,代入(\ref{eq:5-22})并约去$3^t$,得$$9(B(t + 2) + C) -9(B(t + 1) + C) + 2(Bt + C) = t$$,得$B = \dfrac{1}{2}$,$C = \dfrac{9}{4}$,对应的$y^{*(2)}_t = (\dfrac{1}{2} t + \dfrac{9}{4}) 3^t$,再由定理\ref{theorem:5-4}及定理\ref{theorem:5-3}得原方程的通解$$y_t = Y_t + y^{*(1)}_t + y^{*(2)}_t = c_1 + c_2 2^t - t + (\dfrac{1}{2}t + \dfrac{9}{4}) 3^t$$。

		\section{线性差分方程在经济上的应用}

		\subsection{存贷模型}

		\example 某人向银行贷款购房,贷款$A_0$(万元),月息$r$,分$n$个月归还,每月归还贷款数相同,为$A$(万元)。试建立每月应向银行归还$A$(万元)依赖于$n$的计算公式。若$A_0 = 60$(万元),$r = 0.006525$(即年息$7.8390 \%$,折合月息$r = 0.6525 \%$),$n = 180$(即$15$年),该人夫妻每月有$6000$(元)结余可供还贷,问该人是否有能力向银行申请这笔按揭贷款?

		\answer 设至第$t$个月欠款$y_t$(万元),经一个月,本息共计欠贷$(1+r)y_t$(万元),还了$A$(万元),尚欠$$(1+r)y_t - A = y_{t+1}$$,即$$y_{t+1} - (1+r) y_t = -A$$。此为一阶常系数线性差分方程,解之,得通解为$$y_t = c(1+r)^t + \frac{A}{r}$$。初始条件为$y_0 = A_0$,有$A_0 = c + \dfrac{A}{r}$,所以$c = A_0 - \dfrac{A}{r}$,从而得$$y_t = (A_0 - \frac{A}{r})(1+r)^t + \frac{A}{r}$$。$t = n$时,$y_t = 0$,有$$0 = (A_0 - \frac{A}{r})(1+r)^n + \frac{A}{r}$$,求得
		\begin{equation}\label{eq:5-23}
			A = \frac{A_0 r (1+r)^n}{(1+r)^n - 1}
		\end{equation}
		以$A_0 = 60$(万元),$r = 0.006525$,$n = 180$代入上式计算之,$$A = \frac{60 \times 0.006525 \times (1.006525) ^{180}}{(1.006526)^{180} - 1} \approx 0.567528 \text{(万元)} = 5675.18 \text{(元)}$$,每月应还贷$5675.18$元,他们有能力按期偿还贷款。

		有兴趣的读者可以顺便计算一下,$180$个月他们共还贷$1021532.4$元,$15$年共付利息$42.15324$万元。

		\example 设银行存款的年利率为$r=0.05$,并依年复利计算,某基金会希望通过一次性存入$A$万元,实现第一年提取$19$万元,第二年提取$28$万元,\ldots,第$n$年提取($10 + 9n$)万元,并能按此规律一直提取下去,问$A$至少应为多少万元?

		\answer 设第$n$年初,银行还有存款余额为$y_n$万元,则$$y_n(1+r) - (10 + 9n) = y_{n+1}$$即$$y_{n+1} - (1+r)y_n = -(10 + 9n)$$,此为一阶常系数线性差分方程,解之,得通解为$$y_n = c(1+r)^n + \frac{9}{r^2} + \frac{10}{r} + \frac{9n}{r}$$。初始条件为$y_1 = A$,从而知$$A = c(1+r) + \frac{9}{r^2} + \frac{10}{r} + \frac{9}{r}$$,得$c = (A - \dfrac{9}{r^2} - \dfrac{10}{r} - \dfrac{9}{r}) / (1+r)$,所以特解为$$y_n = (1+r)^{n - 1}(A - \frac{9}{r^2} - \frac{10}{r} - \frac{9}{r}) + \frac{9}{r^2} + \frac{10}{r} + \frac{9n}{r}$$。要使得对一切$n$,$y_n \geqslant 0$,所以其充分必要条件是$$A \geqslant \frac{9}{r^2} + \frac{10}{r} + \frac{9}{r}$$。以$r = 0.05$代入计算之,$A \geqslant 3980$(万元),所以一次性存入至少$3980$万元,这样就可按规定要求一直提取下去。

		\subsection{消费模型}

		\example 卡恩(Kahn)提出下述消费模型:设$y_t$和$C_t$分别是时期$t$的国民收入和消费,$I$是每期相同的固定投资,它们满足关系
		\begin{displaymath}
			\begin{cases}
				y_t = C_t + I \\	
				C_t = \alpha y_{t-1} + \beta
			\end{cases}
		\end{displaymath}
		其中常数$0 < \alpha < 1$,$\beta > 0$。已知初始期的国民收入为$y_0$,求$y_t$及$C_t$。

		\answer 由$y_t = C_t + I = \alpha y_{t-1} + \beta + I$,此为一阶常系数差分方程,解之得通解$$y_{t-1} = c \alpha ^t + \frac{\beta + I}{1 - \alpha}$$,或写成$y_t = c \alpha ^{t+1} + \dfrac{\beta + I}{1 - \alpha} = \bar c \alpha ^t + \dfrac{\beta + I}{1 - \alpha}$,其中$c$,$\bar c$为任意常数。再由初始条件,得$$y_0 =\bar c + \frac{\beta + I}{1 - \alpha} \text{,} \bar c = y_0 - \frac{\beta + I}{1 - \alpha}$$,从而得$$y_t = \left(y_0 - \frac{\beta + I}{1 - \alpha}\right) \alpha ^t + \frac{\beta + I}{1 - \alpha}$$,于是有$C_t = \left(y_0 - \dfrac{\beta + I}{1 - \alpha}\right) \alpha ^t + \dfrac{\alpha I + \beta}{1 - \alpha}$。可见,若按此模型,则当$t \to +\infty$时,$y_t \to \dfrac{\beta + I}{1 - \alpha}$,$C_t \to \dfrac{\alpha I + \beta}{1 - \alpha}$

		\subsection{需求量、供应量与平衡价格问题的模型}

		在微分方程中,由供、求关系及价格对时间的瞬时变化率,建立起价格$p(t)$所满足的微分方程并求解,但价格的变化,一般不是瞬时的,而是有时间段的。例如在农业生产中往往是一年为一个时间段,在这种情形中,用差分方程来描述可能更合适。下面就是这种类型的一个模型。

		\example 在农业生产中,设$t$时期某产品的价格$P_t$决定着生产者在下一时期愿意提供市场的产量$S_{t+1}$,且$P_{t}$还决定着本时期该产品的需求量$D_t$,它们的关系为$$D_t = a - bP_t \text{,} S_{t+1} = -\alpha + \beta P_t$$,其中$a$,$b$,$\alpha$,$\beta$均为正常数。又设每一时期$t$,市场的供需总保持平衡:$S_t = D_t$,并设$P_t | _{t=0} = P_0$,求价格随$t$的变化规律$P_t$;并问当$a$,$b$,$\alpha$,$\beta$满足什么条件时,价格趋于稳定。

		\answer 由所给条件,容易建立方程:$$-\alpha + \beta P_t = a - bP_{t+1}$$,即$$P_{t+1} + \frac{\beta}{b}P_t = \frac{a + \alpha}{b}$$。它的通解为$$P_t = c\left(-\frac{\beta}{b}\right) ^t + \frac{a + \alpha}{b + \beta}$$。再由初始条件$P_t | _{t=0} = P_0$,得特解$$P_t = \left(P_0 - \frac{a + \alpha}{b + \beta}\right)\left(-\frac{\beta}{b}\right)^t + \frac{a + \alpha}{b + \beta}$$。显然,当且仅当$|-\dfrac{\beta}{b}| < 1$,即$\beta < b$时,$$\lim _{t \to +\infty}P_t = \frac{a + \alpha}{b + \beta}$$。即当且仅当常数$\beta < b$时,价格$P_t$最终趋于稳定,稳定价格为$\dfrac{a + \alpha}{b + \beta}$。
