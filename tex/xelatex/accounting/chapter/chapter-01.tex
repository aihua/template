\chapter{绪论}
	{\heiti 学习目标}

	\begin{enumerate}
		\item 说明会计的含义;
		\item 了解会计信息的使用者及其需要;
		\item 区分财务会计与管理会计;
		\item 掌握会计信息应达到的质量标准,以及会计规范及其必要性;
		\item 了解会计职业的情况。
	\end{enumerate}

	\section{会计的含义}
	
	什么是会计?会计是一种通用的商业语言。作为商业语言,会计是现代经济生活中一种重要的传递信息、进行交流的工具。运用会计语言对特定对象(如一个企业的经济活动)的描述为人们了解这一对象提供了重要信息,是重要的经济决策与管理的依据。会计作为一种商业语言的作用是逐渐形成的,并且随者社会经济的发展而发展,其作用越来越重要。
	
	会计很早就存在了,它一开始只是运用记数的方法,履行简单地对生产活动数量方面的记录功能,以便管理财物、安排生产。在商业活动尚不发达的时期,会计的运用主要与政府的财政活动特别是与税收有关。在我国的周朝以及古巴比伦和古埃及等都已有进行会计记录的官员,对会计记录的审计也已出现。约从13世纪开始,在地中海沿岸的一些城市,手工业和商业日益发展,对这些商业经营活动的管理显得十分重要,会计方法也随之发展。现代会计的一个重要方法——复式记账就是在那一时期出现的,会计作为一种现代意义上的商业语言也是从那时开始的。从17世纪末的工业革命开始,大工业取代了手工作坊,企业规模扩大,对管理的要求提高。人们需要对生产过程进行更加严密的组织、需要筹集大量的资金、销售产品、支付货款和银行贷款等等。所有这些都需要运用会计来进行及时的记录、分析和报告。而由于企业规模的扩大和企业组织形式的发展导致投资者与管理者相分离,又有了运用会计向投资者报告企业管理者履行职责情况的必要。20世纪以来,科学技术的进步、金融市场的发展、跨国公司的成长、税务管理的复杂等等,使得从宏观管理到微观管理、从组织到个人,人们对会计信息的需求有了前所未有的提高,会计信息在决策与管理中有了不可或缺的地位。
	
	就这样,随着社会和经济的发展,对会计的运用从反映生产经营活动的有关事项到向投资者报告管理者的履职情况,最后发展到向有关各方提供决策信息,人们是通过对会计信息需求的深入、细致的考察来使会计不断适应环境的快速变化并使之得到发展的。
	
	会计作为一种商业语言,它传递的是财务信息,即以货币作为计量尺度的数量信息。会计信息反映的是一个经济实体的经济事项以及它们对该实体的财务状况和财务、经营活动业绩的影响。尽管人们对于将会计视为一种商业语言的理解疑义较少,但是所提出的会计定义却各种各样。有人定义会计是一个信息系统,有人认为会计是一项管理活动,等等。但是,会计是“确认、计量和报告经济信息的过程”这一会计含义的表述还是得到了较大的共识。“确认、计量和报告”是对处理和传输会计信息过程的归纳。简单地说:确认涉及的是什么活动及其影响应该记录、如何记录(应记作什么会计要素,这涉及分类问题)以及何时记录的问题,计量涉及的是确认的对象应如何使用货币等度量单位来衡量的问题,或者说是如何合理地用货币表述确认对象的问报告是向使用者传输会计信息的过程,就是提供已确认和计量的结果,作出分析和解释,并以适当的形式,清晰和及时地提供给使用者,会计的确认、计量和报告这几个过程将在本书后面的章节中进行详细介绍。

	\section{会计信息使用者}

	会计信息使用者因其所作的决策不同,对信息的需求也不同。就一个企业或组织来说,会计信息的使用者可分内部和外部两类。下面以企业为例进行讨论。

		\subsection{内部使用者}
		
		企业内部的使用者是指企业的管理者。企业的管理者包括:公司董事会成员、企业的行政长官(如总经理)、企业的战略规划、财务、供应、营销、人事、研发、技术、物流等行政部门的管理人员。管理者在实施管理时,需要很多会计信息。比如,一个公司经理可能会问:现金是否足够支付公司的借款?客户是否能及时支付货款?每一单位产品的制造成本是多少?成本有没有超过预算?今年公司能否负担得起员工工资的上涨?哪一种产品最赚钱?为扩大业务需要借多少钱?等等。
		
		为了帮助管理者回答上述一系列问题,会计就要提供内部报告。内部报告的内容可涉及内部的决策、计划和控制等管理的不同过程。

		\begin{enumerate}
			\item 决策过程 \\
						从决策过程看,会计提供的信息涉及多个方面:
						\begin{enumerate}
							\item[(1)] 生产决策。在生产什么、生产多少以及如何组织生产等问题上,会计可从各产品对企业利润的贡献、零部件外购和自制的不同成本、产品或作业的成本构成以及降低成本的途径等等方面提供信息。
							\item[(2)] 营销决策。在确定销售价格、销售策略等方面,会计可提供的信息有:从企业的成本、利润等角度评价产品价格的合理性、说明不同营销方式下的销售费用、分析不同信用条件下的销售额以及坏账和收账费用等。
							\item[(3)] 投资决策。会计可从投资所需的资金额、未来能带来的利润或现金流量、资本成本等因素说明投资是否合算。
							\item[(4)] 筹资决策。会计可从企业扩大经营、增加新业务等所需资金的量、不同筹资方式的融资成本和它们对企业未来现金流量的影响等方面分析筹资的金额以及筹资方式的合理性等。
						\end{enumerate}
			\item 计划和控制过程 \\
						从计划和控制过程看,会计的一个重要作用体现在编制预算上,预算是计划的重要形式,是用货币量化的行动指南。而且,通过预算可使各部门在企业目标下得以沟通和协调,预算也为业绩评价提供了一个标准。它包括销售、生产、原材料、人工费用、制造费用、销售和管理费用等预算以及现金预算、资本预算等。预算覆盖了某一特定时期(通常是一年)整个企业的经营和财务活动。控制就是使事情按计划进行。会计信息在控制过程中的作用也是明显的。会计信息可反映企业各部门的工作业绩和预算执行情况,会计可通过实际结果与预算标准的比较找出两者的差异,通过差异分析找出产生差异的原因,从而才能提出改进工作或计划的措施。另外,会计的这些信息也为企业确定职工工资水平、岗位设置和采取其他奖惩措施提供了依据。
		\end{enumerate}

		\subsection{外部使用者}

		企业会计信息的外部使用者,是相对于企业管理层而言的。主要的外部信息使用者包括:投资者,债权人,政府或监管部门以及其他与企业相关的人。

		\begin{enumerate}
			\item 投资者 \\
						投资者就是企业的所有者,比如公司的股东。所有者之所以属于企业“外部”信息使用者,是由于现代企业使所有权和管理权相分离,所有者可不直接参与企业的管理。这不但使所有者了解企业情况的途径不同于企业的管理者,而且导致了在会计上将企业视为独立于所有者的实体。投资者,包括潜在的投资者,主要关心的是企业过去的经营是否成功以及将来可能获利的情况。他们在投资之前,要判断自己投资的获利前景和风险因素,将投资对象进行比较,以决定是否投资;资本投入企业后,他们还要继续关注企业的盈利能力、利润分配和财务状况,以决定是否再投资或撤资。

						上市公司由于其股票是在证券市场上公开发行的,其投资者就是一般公众。投资者为购买、持有或转让股票,有时可以咨询专家来作决策,而这些专家就要利用会计信息了解企业情况、作出股票投资的分析评价。
			\item 债权人 \\
						债权人包括银行等金融机构、企业债券的持有者以及被企业欠钱的其他单位和个人。债权人主要关心的是企业能否按期支付利息和归还借款。所以,债权人要根据会计信息考察企业资产的流动性和现金流量,评估信用风险,作出是否借款和如何保证借款安全的决策。
			\item 政府或监管部门 \\
						企业应依法缴纳各种税金,税务部门不但要求会计提供各种税务专用报表,而且还要对相关的会计信息进行审计,以确定企业应纳税额和企业依法纳税的情况。
						此外,其他政府或监管部门,如财政部门、证券监督管理委员会等,为了各自的职责需要企业的会计信息。此外,为了保证会计信息的作用和保护信息使用者的权利,也需要依赖会计信息及其披露方面的情况来制定有关会计的法律法规。
			\item 其他方面 \\
						其他与企业利益相关方包括范围比较广,主要有以下几个方面。
						\begin{enumerate}
							\item[(1)] 供货方。他们最关注的是企业的支付能力,根据会计信息判断企业能否根据合同按时支付货款(有时供货方也是企业的债权人),以决定是否供货(或提供服务)。
							\item[(2)] 客户。为决定是否继续使用企业产品或服务,客户要判断企业供货的长期可靠性,就笛要了解企业的财务状况和获利能力是否能保证企业正常经营、继续保质保量地提供产品。
							\item[(3)] 职工。职工通过会计信息了解企业的盈利水平和支付能力,判断企业是否能按时支付工资、是否有能力搞好劳动保护、提供较好的工作环境、增加工资、保障就业等。
							\item[(4)] 咨询中介机构。会计师事务所、资产评估机构、其他咨询或中介机构的专家也需要利用会计信息为各自的委托者对企业及其会计信息进行审计、分析和评价。
						\end{enumerate}
						为满足企业外部使用者的需要,会计定期提供具有规定格式和内容的财务报告。按国际会计准则理事会(IASB)的说法,财务报告的主要内容应包括:
						\begin{enumerate}
							\item[(1)] 决定何时购入、持有或出售一项权益性投资(如股票);
							\item[(2)] 评价管理当局的保管或受托责任;
							\item[(3)] 评价企业向其职工支付工资和提供其他利益的能力;
							\item[(4)] 评价对企业贷款的保险程度;
							\item[(5)] 决定税务政策;
							\item[(6)] 决定可分配利润和股利;
							\item[(7)] 编制和使用国民收入统计指标;
							\item[(8)] 管理企业的活动。
						\end{enumerate}
		\end{enumerate}

		\section{财务会计与管理会计}

		会计按其服务对象的不同,可分为财务会计和管理会计。

		财务会计是为企业外部信息使用者服务的会计。它定期向外提供以会计报表为主要内容的财务报告。我国《企业会计准则——基本准则》规定:“财务会计报告的目标是向财务会计报告使用者提供与企业财务状况、经营成果和现金流量等有关的会计信息,反映企业管理层受托责任履行情况,有助于财务会计报告使用者作出经济决策。”

		管理会计是为企业内部管理服务的会计。它根据需要为管理者提供内部报告。

		\subsection{财务会计与管理会计的区别}

		财务会计与管理会计的根本区别就是服务对象的不同,由此还产生了其他很多不同,主要有以下几个方面。
		
		\begin{enumerate}
			\item[(一)] 信息处理和报告的限制
				\begin{enumerate}
					\item[1、] 财务会计 \\
						从上一节可知,会计信息的外部使用者涉及很多方面,他们分别都有自己关心的问题,但财务会计往往不是一对一提供报告的。在大多数情况下,使用者一般得不到专门 针对自己解决具体问题的信息,只能接受企业提供给外部的同一财务报告。所以,外部使用者如果不掌握报告编制的基本规则就很难理解这些信息的含义。而如果每一企业 都有自己编制报告的规则,那么,当信息使用者进行决策需要将不同企业进行比较时, 他必须掌握各企业的不同规则,否则,就无法比较。比如,某个公司告诉你它今年赚了$1,000$万元利润,如果你不知道它把什么称为利润、今年这$1,000$万元又是怎么确定的, 这一信息对你的用处就不大。如果另一公司说它今年的利润是$2,000$万元,当你不知道 这两家公司确定利润的口径是否一致时,你又怎么能判断这$2,000$万的利润就一定比那$1,000$万多呢?而且,还有最重要的一点,就是为保证会计信息的公允性,需要有统一的标准来规范会计信息的处理和报告。
						
						因此,财务会计为满足外部使用者的一般性需要,其信息的处理与报告应遵循权威机构制定的原则或统一的制度,以保证信息的公允、可比和便于使用。
						
						对财务会计的这种限制严格而周到,可涉及会计的处理程序、会计账证表的格式与保管、会计确认与计量的原则、会计方法的选择原则、财务报告的具体内容和形式等。这些原则或制度将在本章的“会计规范”一节以及以后相关章节中再具体讨论。
					\item[2、] 管理会计 \\
						由于管理会计信息是企业自己内部使用的,管理者需要什么信息就可让会计部门提供,而且各企业都有自身特有的管理问题,所以,管理会计没有权威机构制定的强制性的信息处理和报告原则。会计处理程序、内部报告的形式和内容等可以灵活多样。当然,管理会计也要受成本效益等原则的约束。
				\end{enumerate}
			\item[(二)] 信息的类型
				\item[1、] 财务会计 \\
					财务会计为保障企业外部使用者的权利,其信息是客观的,亦即是以可验证的事实为基础的。从时间上看,为了保证客观或可验证,财务会计信息往往只反映企业已经发生的事项,是对过去的总结。财务会计信息基本都是以货币表述的财务信息,如我国会计报表中的各项目就是以“元”为单位的。
				\item[2、] 管理会计 \\
					管理会计不但提供客观信息,而且也有必要提供主观信息,如对将来发生事情及其影响因素和可能性的推测。从时间上看,管理会计尽管也有过去信息,但它强调的是控制现在和预测未来,这与管理会计紧随各管理环节提供服务有关。管理会计除了财务信息之外也有一些非财务信息,如以实物量、工时等计量的信息。
					
					总之,财务会计信息基本是客观可验证的、过去的、财务性的,而管理会计信息主客观兼容、注重未来、财务与非财务并存。
			\item[(三)] 信息的类型
				\item[1、] 财务会计 \\
					财务会计信息的内容往往是将企业作为一个整体,综合反映一个企业的整体财务状况、现金流量和经营成果。
				\item[2、] 管理会计 \\
					管理会计信息既反映整体又反映局部,在局部方面可详细提供企业内部各部门、产品、作业、个人等情况,局部信息不但能帮助某一具体决策问题的解决,也有助于落实计划、评价业绩与考核等。
			\item[(四)] 信息的时期跨度
				\item[1、] 财务会计 \\
					财务会计的报告期有严格规定,通常以一年为报告期,也可以按月、季、半年提供中期报告。报告期比较固定,财务会计通常较严格地按报告期分隔各项连续的经营活动,按报告期提供一个企业全部活动在该期内引起的经营成果的信息。
				\item[2、] 管理会计 \\
					管理会计可灵活地根据管理要求提供不同时间跨度的信息,比如可按周、按日甚至按小时落实预算、反映资金调度情况、报告某方面的工作进展等,也可以按5年、10年、20年反映某一项活动从开始到结束的整个过程及其影响。
		\end{enumerate}

		\subsection{财务会计与管理会计的联系}
			尽管财务会计与管理会计有上述这些区别,但是两者之间的联系也是十分明显的:

			\begin{enumerate}
				\item[(1)] 财务会计和管理会计是会计的两个部分,在企业的管理信息系统中它们同属于会计信息系统。财务会计和管理会计的最终目标都是为决策者提供信息,信息描述的对象都是与一个企业的经济事项及其影响有关。
				\item[(2)] 管理会计信息常常是在财务会计信息基础上生成的,财务会计和管理会计的数据基础是相同的,这些会计记录是为最终编制财务会计报表而准备的,经过进一步加工处理,就成了内部管理需要的信息。
				\item[(3)] 有时企业内部与外部决策者所需会计信息是一样的。由于企业内部与外部决策者需要同样的会计信息,所以,企业管理者可以直接利用财务会计信息。而由于“内外有别”,使得外部使用者不能获得属于管理会计的信息。
			\end{enumerate}
			
			本书第二章至第八章主要讨论财务会计,笫九章介绍管理会计。

			\section{会计规范}

			如前所述,会计信息是很多人进行经济决策的重要依据,能对信息使用者的利益产 生重大影响。所以,为保障信息使用者的权利,使会计信息尤其是财务会计信息公允、 可比、符合使用者的需要,建立会计规范十分必要。

			\subsection{我国企业会计规范体系概述}
				我国现行的会计规范是一个由国家法律、行政法规、部门规章和地方性法规等有机 组成的体系。图1-1就是我国与企业财务会计有关的规范体系的简略图。

				图1-1 我国企业会计规范体系示意

				我国的会计规范具体有这样几个层次:

				\begin{enumerate}
					\item[(一)] 会计法律 \\
						国家法律是由国家最高权力机构——全国人民代表大会及其常务委员会制定的,属这一层次的会计法律是《中华人民共和国会计法》(简称《会计法》),它是会计工作的基本法,是其他层次会计法规的依据。在会计规范体系中《会计法》居于最高层次。
						
						《会计法》由全国人大于1985年颁布,并经1993年和1999年两次修订,《会计法》共分七章五十二条,主要在会计核算、会计监督、会计机构、会计人员、法律责任等方面作出了规定。《会计法》划分了会计工作的管理权限,确定了会计工作的地位和作用,规定了会计机构和会计人员的主要职责,明确了会计人员的职权和行政职权的法律保障。同时还明确了单位负责人和会计人员履行职责的法律责任。另外,《会计法》还专门对公司、企业的会计核算作了特别规定。
						
						此外,我国还有专门针对注册会计师及其执业的法律:《注册会计师法》。
					\item[(二)] 会计行政法规 \\
						行政法规是由国家最高行政机关——国务院制定的。会计行政法规是会计法律的具体化和某个方面的补充。在我国现行会计规范中,属于会计行政法规的主要有《总会计师条例》、《企业财务会计报告条例》等。
						
						《企业财务会计报告条例》由国务院发布于2000年6月21日,自2001年1月1日施行。条例是依据《会计法》制定的。条例共分六章四十六条,对企业(包括公司)编制和对外提供财务会计报告作了规定,主要涉及财务会计报告的构成、编制、对外提供、法律责任等几个方面内容。条例严禁提供虚假或隐瞒重要事实的财务会计报告,明确了应由企业负责人对本企业财务报告的真实性和完整性负责,还规定了注册会计师审计财务报告的法律依据。条例具体规定了财务报告的期间和构成内容、基本会计报表的项目分类、定义及列示、附表附注和财务情况说明书的具体内容,条例还规定了报告的编制基础、编制依据、编制原则和方法等,条例还对财务报告的封面、签章、报告的提供对象、报告的审计等作了规定。
					\item[(三)] 会计部门规章 \\
						部门规章是由国务院的各部委制定的,会计的部门规章主要由国家主管会计工作的行政部门——财政部制定。《会计法》规定国家实行统一的会计制度。国家统一的会计制度由国务院财政部门根据本法制定并公布。而会计的部门规章就属于“国家统一的会计制度”,是依据会计法律、行政法规制定的。国家统一的会计制度具体可包括会计核算、会计监督、会计机构和会计人员管理、会计工作管理等方面的制度。
						
						会计核算方面的部门规章,从企业会计看,主要是企业会计基本准则。企业会计基本准则在整个企业会计准则体系中起着统驭作用,是建立在财务会计的概念基础上的。
						
						《企业会计准则——基本会计准则》,是我国财政部于2006年对1992年发布的《企业会计准则》进行修订后公布的。基本准则共有十一章五十条,它规范了包括财务报告目标、会计基本假设、会计信息质量要求、会计要素的定义及其确认、计量原则、财务报告等在内的基本问题,是会计准则制定的出发点,是制定具体准则的基础。
						
						会计监督制度:对会计监督的规定散见于相关的会计制度中,比如《会计基础工作规范》中就有关于会计监督的内容。
						
						会计机构和会计人员管理制度:有《会计从业资格管理办法》《会计人员继续教育暂行规定》等。
						
						会计工作管理制度:有《会计档案管理办法》、《会计电算化管理办法》、《代理记账管理办法》等。
					\item[(四)] 规范性文件 \\
						企业会计准则体系包括基本准则、具体准则和应用指南,后两者是规范性文件,是 在基本准则基础上的具体化,以便在实务中可操作。
						\begin{enumerate}
							\item[1、] 企业会计具体准则 \\
								企业会计具体准则是根据会计法和企业会计基本准则以及其他相关法律法规而制定的。它具体规范了企业会计确认、计量和报告行为,以保证会计信息质量。我国目前的企业会计具体准则共有38项。

							\item[2、] 企业会计准则应用指南 \\
								会计准则应用指南是企业会计准则体系的重要组成部分。企业会计准则应用指南主要是对企业会计准则中的重点、难点和关键点作出解释性规定和说明,并且还根据具体准则中涉及确认和计量的要求,规定了会计科目及其主要账务处理。
								
								除上述之外,我国还有涉及会计的其他法律法规,比如:我国证券市场实现证券信息持续公开制度,中国证券监督管理委员会根据《公司法》、《证券法》发布关于信息披露的部门规章,主要有:《公开发行证券的公司信息披露编报规则》、《公开发行证券的公司信息披露内容与格式准则》等,它们对会计信息的披露作出了详细的规定。本书将在笫七章中介绍有关内容。另外,如《审计法》、《证券法》、《公司法》、《税收征收管理法》,各种税法、各种企业以及银行、保险法等相关法律法规、部门规章都有关于会计方面的规定。
								
								一直以来,世界各国基本都有自己的会计规范,但随着全球经济的发展和国际上财务会计信息可比性的要求,会计准则越来越趋于国际化。目前,世界上最具有国际性的会计准则应是由国际会计准则理事会(International Accounting Standards Board,IASB)发布的《国际财务报告准则》(International Financial Reporting Standards,IFRS),2003年之前叫《国际会计准则》(International Accounting Standards,IAS)。我国2006年发布的会计准则已与国际会计准则趋同。

						\end{enumerate}
				\end{enumerate}

				\subsection{对会计信息的质量要求}
				会计信息质量对信息使用者的重要性是毋庸置疑的,在上述的我国会计规范中,对于禁止虚假财务会计信息、要求企业对外提供真实和完整信息的内容占有重要地位。在《企业会计准则——基本准则》中还对财务会计信息质量提出了具体要求,这实际上是对会计信息产品提出的“质量标准”。
					
				在《企业会计准则——基本准则》的第二章“会计信息质量要求”中,有如下具体规定。
				\begin{enumerate}
					\item[(一)] 可靠性 \\
						“企业应当以实际发生的交易或事项为依据进行会计确认、计量和报告,如实反映符合确认和计量要求的各项会计要素及其他相关信息,保证会计信息真实可靠、内容完整”。
						
						可靠性主要有三层意思:
						
						一是会计信息应以实际发生的事实为根据,不能虚构、歪曲或隐瞒事实,并要能经受验证,即不同的人对同一事项采用相同的计量方法应当得出相同的结果。
						
						二是在提供会计信息时,要中立而不带有偏向,即不能通过使信息达到某种想要达到的结果,来影响会计信息使用者的特定决策。会计信息当然会影响决策,但它的这种影响应对所有人都一样。
						
						三是会计信息应能够真实反映其意欲反映的交易或事项,这要求会计选择正确的计量或确认方法。
						
						会计信息有时不是由于造假、有偏向,而是由于确认、计量方法的选择等技术上的原因,使其出现错误或不能完全真实地反映客观实际,这就影响了信息的可靠性。
					\item[(二)] 相关性 \\
						“企业提供的会计信息应当与财务会计报告使用者的经济决策需要相关,有助于财务会计报告使用者对企业过去、现在或者未来的情况作出评价或者预测。”
						
						相关性是会计目标本身所要求的,会计提供的信息应适合使用者的需要。就是说,会计信息应与使用者的决策相关,具有影响决策的能力。会计信息与决策的相关性表现在能影响信息使用者对经济事项的预测(信息的预测价值)或者能影响信息使用者进一步确信或修正自己的预测(信息的反馈价值)。

						可靠性和相关性是两个主要的会计信息质量要求,既可靠又相关是会计信息质量应达到的标准,是保证会计信息有用性的条件。但在会计实务中有时会遇上两难的境地,不得不在两者之间进行权衡。

					\item[(三)] 可理解性 \\
						“企业提供的会计信息应当清晰明了,便于财务会计报告使用者理解和使用。”
						
						这里所谓的“清晰明了”是针对会计信息使用者的,即从信息使用者看来信息是清楚的、容易理解的。这涉及两个方面:一是会计记录与报告的表达方式;二是会计信息使用者本身所具备的理解力和知识,对此《国际会计准则》是如此假定的:“使用者对商业和经济活动以及会计有恰当的了解并且愿意花费适当的精力去研究信息。然而,有些关于复杂事项的信息由于它们与使用者作经济决策的需要相关而应包括在财务报表之中,不能仅仅因为这些信息对于某些使用者来说过于难以理解而将它们排除在财务报表之外。”
					\item[(四)] 可比性 \\
						“企业提供的会计信息应当具有可比性。同一企业不同时期发生的相同或者相似的交易或者事项,应当采用一致的会计政策,不得随意变更。确需变更,应当在附注中说明。不同企业发生的相同或者相似的交易或者事项,应当采用规定的会计政策,确保会计信息口径一致、相互可比。”

						可比性包括企业间的会计信息的可比性,这是一种横向可比性;也包括一个企业所选定的方法在各个时期延续使用,这是会计信息在不同时期的纵向可比性,也称一致性。实际上,可比性往往是所有定量信息的质量要求。
					\item[(五)] 实质重于形式 \\
						“企业应当按照交易或者事项的经济实质进行会计确认、计量和报告,不应仅以交易或者事项的法律形式为依据。”
						
						“实质重于形式”本身不是一种质量特性,但它体现了可靠性的要求。在这一原则下,当会计事项出现了法律形式与经济实质不符的情况时,会计应按经济实质反映。比如,企业将产品出售给买方,售价$10$万元,产品的所有权凭证已转移、实物也已交付给买方、买方支付的货款也已收到,但合同规定$3$年后企业应以$12$万元将产品从买方处购回。这一交易从法律形式看,产品所有权已转移,企业是在销售产品。而实质上这是企业以产品作抵押向买方借款。所以,在会计上不将这笔交易的$10$万元确认为销售收入,而是作为企业的负债。由此可见,实质重于形式是为了能“透过现象看本质”,更加合理地反映企业的经济活动情况。
					\item[(六)] 重要性 \\
						“企业提供的会计信息应当反映与企业财务状况、经营成果和现金流量有关的所有重要交易或者事项。”
						
						重要性原则的提出体现了在信息所产生的成本与效益之间的权衡。为了使效益大于成本,次要的会计事项就可适当简化。实际上,重要性是保证相关性的一个取舍标准。当然,实现这一原则的关键是如何判断交易或事项的重要程度。
					\item[(七)] 谨慎性 \\
						“企业对交易或者事项进行会计确认、计量和报告应当保持应有的谨慎,不应高估资产或者收益、低估负债或者费用。”
						
						谨慎性原则本身也不是一种质量特性,它是保证会计信息可靠性的一个要求。这一原则的提出是由于会计所应反映的事项或交易存在不确定性。所以,要求会计对于具有估计性质的事项或交易应当合理预计可能发生的费用或负债,但不预计或少预计可能带来的收入或资产。比如,企业购入存货,在存货的持有期间,若存货价格下跌使企业的可变现净值低于购买成本之时,企业就将这一跌价损失反映出来;若存货价格上涨,在存货持有期间,却不反映因此而产生的涨价收益。这就是谨慎原则的体现。事实上,只要存货没有实际出售,无论其价格是下跌还是上涨,所产生的损失或收益都只是一种预计而已。这种预计应当建立在尽可能合理的基础上。

					\item[(八)] 及时性 \\
						“企业对于已经发生的交易或者事项,应当及时进行会计确认、计量和报告,不得提前或者延后。”
						
						及时性就是应在交易或事项发生后及时收集、整理原始单据、及时进行会计处理并按规定时间提供财务报告。提前或延后都会影响会计信息质量。如果在证明有关交易或事项发生的单据还不齐全时就提前进行确认、计量,就会影响会计信息的可靠性;如果不及时报告、拖延了时间,就会影响会计信息的相关性,也就是说,会计信息应在失去其影响决策的能力之前提供才是及时的。所以,及时提供信息是保证信息相关性的一个条件。

						关于会计信息的质量要求,在国外的会计规范中也有规定,比如《国际会计准则》提出了四项主要质量特征:可理解性、相关性(包括重要性)、可靠性(包括如实反映、实质重于形式;中立性、审慎;完整性)、可比性等。

				\end{enumerate}

			\section{会计职业}
				现代会计职业在$19$世纪中期的英国兴起,如今它已成为社会经济生活中不可缺少的、在世界上拥有庞大从业人员的职业。

				\subsection{会计职业的分类}

				从本章前面的讨论可知,会计按信息使用者不同可分为财务会计和管理会计,这两方面是会计专业人员的主要工作领域,但会计从业人员还可以运用他们的会计专业知识和技能从事审计、咨询等方面的工作。会计职业常常按会计人员所工作的不同社会部门分类,一般可分为向公众提供服务的“公共会计”和为某一组织服务的“企业会计”、 “非营利组织会计”。

				\begin{enumerate}
					\item[(一)] 公共会计 \\
						公共会计是为不同的客户服务的会计师,称他们为“公共会计”,是因为他们为公众提供服务,己执业,不受雇于别的企业或单位。就如自己开业的律师事务所或诊所的律师和医生。

						在我国,公共会计就是注册会计师,根据我国《注册会计师法》(1993年颁布)的规定,注册会计师的业务范围主要包括:审计业务和会计咨询服务。

						\begin{enumerate}
							\item[1、] 审计业务 \\
								注册会计师进行的是独立的社会审计。审计业务是注册会计师的最基本业务,就是接受客户委托进行查账和验证,主要内容是:审查企业会计记录和会计报表,出具审计报告,提出审计意见,证明企业会计报表和财务报告的编制是否符合会计准则或制度、是否客观真实;验证企业投入资本,出具验资报告;参与办理企业合并、分立、清算事宜中的审计业务,出具有关报告等;参与办理企业的经济纠纷,协助鉴别经济案件证据等。

							\item[2、] 会计咨询、服务业务 \\
								为企业设计会计制度,担任会计顾问,提供会计、财务、税务或经济管理方面的咨询等;为企业代理纳税审报;代办申请注册,协助拟订合同、章程和其他经济文件等。

								我国有注册会计师团体——“注册会计师协会”,分为全国和省级两级组织,在国家和省级财政部门监督、指导下实行行业的自律管理。

								在世界各国基本都有公共会计,如美国的注册会计师(Certified Public Accountant,CPA)、英国、加拿大等国的特许会计师(Chartered Accountant,CA)等。

						\end{enumerate}

					\item[(二)] 企业会计 \\
						企业会计是受雇于某个企业或公司的会计人员。每个企业都需要会计,除了规模很小的企业,大多数企业都设有会计部门和专职会计人员。企业会计的工作范围大体上可分为:

						\begin{enumerate}
							\item[1、] 记录日常交易或事项、编制会计报表和提供其他企业内外所需信息。
							\item[2、] 成本会计:确定生产过程中的各项费用、计算产品成本等。成本会计的很多信息为内部管理所用,但也有部分内容向外提供。
							\item[3、] 预算与控制:协助管理者以货币量化企业目标、编制预算,记录、分析和报告内部各部门的预算执行情况。
							\item[4、] 税务会计:税务部门是会计信息的外部使用者之一,但由于财务会计与税法所遵循的原则和规范的对象不同,所以企业会计制度与税法对会计要素的确认、计量的方法也不尽相同。我国目前的会计制度制定的原则是:能与税法保持一致的尽量保持一致,不能一致的就适当分离。这就使得企业会计人员对两者不一致的地方需要进行纳税调整,除向税务机关提交通用的财务会计报表外,还需提供专门的税务资料。另外,会计还可为企业制定税务计划。
							\item[5、] 内部审计:考察企业的财务状况和经营业绩、评价企业管理者的履行职责的情况,旨在加强内部管理和监督、改善经营管理、提高经济效益。我国《审计法》规定,国有企业应设置独立的内部审计机构(审计业务少,可只设专职内部审计人员)。
						\end{enumerate}

					\item[(三)] 非营利组织会计 \\
						非营利组织会计是指受雇于政府机关、事业单位、社会团体等组织的会计人员。由于这些组织的经费主要由财政预算拨给,所以这些组织的会计一般可统称为“预算会计”。预算会计的工作范围主要是对预算经费的收入和使用进行记账和报账,反映这些单位的预算执行情况。而由于现在事业单位已打破了“供给制”模式,事业单位的所有制形式已不仅仅只有“国有”一种、其筹资渠道也不仅仅只有单一的财政拨款,所以事业单位会计信息的外部使用者增多、内部管理因讲求经济效益也提高了对会计信息的要求。因此,事业单位会计工作的内容除记录和报告预算资金收支情况之外,还需要核算经营收入、成本等。

				\end{enumerate}

				\subsection{会计的从业资格}

				\begin{enumerate}
					\item[(一)] 职业道德 \\
						人们从事任何工作都需要职业道德,职业道德是指导人们执业的行为规范。会计提供的是信息服务,会计人员应当诚实、公平、负责、为客户或雇主提供优质服务。
						
						《会计基础工作规范》($1996$年财政部发布)规定了会计人员职业道德:会计人员在会计工作中应当树立良好的职业品质、严谨的工作作风,严守工作纪律,努力提高工作效率和工作质量。会计人员应当热爱本职工作,努力钻研业务,使自己的知识和技能适应所从事工作的要求;应当熟悉财政法律、法规、规章和国家统一会计制度,并结合会计工作进行广泛宣传;应当按照会计法律、法规和国家统一会计制度规定程序和要求进行会计工作,保证所提供的会计信息合法、真实、准确、及时、完整;会计人员办理会计事务应当实事求是、客观公正;应当熟悉本单位的生产经营和业务管理情况,运用掌握的会计信息和会计方法,为改善单位内部管理、提高经济效益服务;应当保守本单位的商业秘密,除法律规定和单位领导人同意外,不能私自向外界提供或者泄露本单位的会计信息。
						
						《注册会计师法》所规定的注册会计师工作规则,实际上也涉及职业道德:注册会计师在执行业务时,应当遵守国家法律、行政法规,以有关协议、合同、章程为依据;恪守公正、客观、实事求是的原则,对所出具报告书内容的正确性、合法性负责;对在执行业务中知悉的商业秘密,应当保密;在执行业务中,发现委托人其会计处理与国家有关规定相抵触、损害报告使用人等他人的利益、报表不实等应在的报告书中明确指出;对委托人示意作不实或者不当作的证明以及其他不合理要求和行为应予以拒绝。
					\item[(二)] 从业资格的取得
						\begin{enumerate}
							\item[1、] 注册会计师 \\
								我国《注册会计师法》规定,取得注册会计师资格应当参加全国统一的考试。具有高等专科以上学校毕业的学历,或者具有会计或者相关专业中级以上技术职称的中国公民,都可以申请参加注册会计师全国统一考试;具有会计或者相关专业高级职称的人员可免予部分科目的考试。考试合格,并具备从事独立审计业务工作$2$年以上、年龄在国家规定的职龄以内等条件的,可向省级注册会计师协会申请注册。准予注册的申请人,由注册会计师协会发给财政部门统一制定的注册会计师证书。注册会计师要执行业务,就必须加人会计师事务所。会计师事务所可以分为负有限责任的法人和负无限责任的合伙制两种。另外,国家机关现职工作人员不得担任注册会计师。 

								\item[2、] 企业和非营利组织会计师 \\
								我国《会计从业资格管理办法》($2005$年颁布)规定,在国家机关、社会团体、企业、公司、事业单位和其他组织从事会计工作的人员,必须取得会计从业资格,持有会计从业资格证书。
								
								遵守会计和其他财经法律、法规,具备良好的道德品质,具备会计专业基础知识和技能的,并具备教育部门认可的中专(含中专)以上会计类专业学历(或学位),自毕业之日起两年内(含两年),只要通过“财经法规与会计职业道德”科目的考试,就可申请取得会计从业资格证书;不具备以上条件的,必须参加会计从业资格考试,从业考试由各省级财政部门统一组织,成绩合格方可申请取得从业资格证书。会计从业资格实行注册登记制度并定期年检资格证书。
								
								取得从业资格是从事会计职业的起点。只有取得会计从业资格的人员,才可根据学历和工作时间等条件,通过考试或评审,取得专业技术资格。取得专业技术资格后,就可参加相应会计专业技术职务的聘任。会计专业技术资格分为:初级资格、中级资格和高级资格。会计专业技术职务分四级:会计员、助理会计师、会计师、高级会计师。
								
								会计专业技术资格考试从$1992$年开始实行全国统考,由财政部、人事部共同负责。目前,考试分为两个级别:初级资格考试(是会计员和助理会计师资格考试)和中级资格考试(是会计师资格考试)。
								
								取得初级资格,并且大专毕业担任会计员职务满$2$年,或者中专毕业担任会计员职务满$4$年,或者不具备规定学历、担任会计员职务满$5$年,单位就可根据有关规定,聘任其为助理会计师。取得初级资格,但不符合上述条件的人员,只可聘任会计员职务。
								
								取得中级资格,并且取得大学专科学历、从事会计工作满$5$年,或者取得大学本科学历、从事会计工作满$4$年,或者取得双学士学位或研究生班毕业、从事会计工作满$2$年,或者取得硕士学位、从事会计工作满$1$年,或者取得博士学位单位就可根据有关规定,聘任其为会计师职务。
								
								高级资格(高级会计师资格)没有全国统一的资格考试,实行考试与评审结合的评价制度。

								此外,所有取得从业资格的人员都必须接受继续教育。继续教育的主要内容包括会计理论与实务、财务与会计法规制度、会计职业道德规范等。继续教育分级别进行,形式多样。
								
								综上所述,会计人员从取得从业资格到获得专业职务的过程大体是:
								
								取得从业资格$\Rightarrow$取得专业技术资格$\Rightarrow$受聘专业职务。
								
								除了技术职务,我国在一些单位实行总会计师制度,《会计法》规定:“国有的和国有资产占控股地位或者主导地位的大、中型企业必须设置总会计师。”总会计师不是技术职称而是行政职务,是主管本单位的会计机构进行会计核算、会计监督工作的负责人。
						\end{enumerate}

				\end{enumerate}

			\section{小结}

			本章为了让读者首先对会计有一个大致的了解,因此讨论了:什么是会计、会计提供的是什么样的信息、谁需要会计信息、由谁提供信息、会计信息作为会计的“产品”应有什么样的质量标准、如何进行市场规范等问题。以后各章内容会使这些问题得到更加深入的展现。
			
			会计是一种商业语言,会计提供的信息是现代经济生活中人们进行经济决策的重要依据。从一个企业的角度看,会计信息的使用者有企业内部的管理者和企业外部的投资者、债权人、税务部门、供应商、客户等与企业有利益关系的各个方面,从而使会计分成为外部提供信息的财务会计和为内部管理服务的管理会计,并造成了财务会计与管理会计的不同特点。财务会计信息要求公允和可比,需要统一的信息处理和报告原则或制度,受到严格规范;而管理会计信息为企业内部使用,就无须外部的统一原则或制度的限制。财务会计信息必须客观、便于验证,内容上主要反映过去;而管理会计提供大量的预测的或主观的信息来满足内部管理的需要。
			
			本章介绍了我国现行的会计规范体系。并专门就会计规范中对会计信息的质量要求作了说明。根据我国会计准则、国际会计准则等,会计信息质量要求或特征主要有四项,即可理解性、可靠性、相关性和可比性。而且这些质量特征并不是并列的,而是有层次的。可理解性说明会计信息是以用户为出发点的,它是信息质量的基础;可靠性和相关性是两个主要的质量要求,既可靠又相关才是有用信息;实质重于形式和谨慎性是可靠性的需要,而重要性是在成本与效益进行权衡时,相关性应坚持的“底线”;可比性和一致性是数量信息的一般质量要求,也是使会计信息有用的要求。
			
			会计职业按会计人员的服务部门分为两大类:为公众服务的注册会计师以及为某一组织服务的企业会计和非营利组织会计。任何会计人员都需要具备会计职业道德,并取得规定的从业资格。 

			{\heiti 思考题}

			\begin{enumerate}
				\item 会计是什么?
				\item 财务会计与管理会计的联系与区别是什么?
				\item 为什么需要会计规范,本章提到的会计规范内容有哪些?
				\item 会计信息最重要的质量特征是什么?
				\item 从事会计职业需要什么样的条件?会计的专业技术资格与专业技术职务有哪些?
			\end{enumerate}
