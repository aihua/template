\chapter{会计要素}
	{\heiti 学习目标}

	\begin{enumerate}
		\item 了解会计的四个基本假设;
		\item 熟练掌握会计要素的概念、资产和权益的关系、资产和权益的变化及其类型、资产负债表的基本内容;
		\item 熟练掌握收入和费用的概念与两者的关系,熟练掌握收入、费用与资产、权益的关系以及利润表的基本内容;
		\item 掌握现金流量的概念,掌握现金流量与资产、权益的关系以及现金流量表的基本内容。
	\end{enumerate}

	本章讨论会计要素以及各会计要素之间的关系。会计要素构成了会计报表的基本内容,是会计对其确认、计量和报告的对象的分类。掌握会计要素的概念和各要素的关系是读懂会计报表、理解会计信息的基础。

	\section{会计的基本假设}
	
	在讨论会计要素之前,先说明会计的基本假设,因为会计的基本假设是会计确认、计量和报告的前提,是对会计核算所处时间、空间环境等所作的合理设定。反映了会计信息内容的基本特征,会计基本假定有四条:会计主体、持续经营、会计分期、货币计量。我国的会计准则对这四个基本假设作了规定。
	
		\subsection{会计主体}

		会计主体,是指企业会计确认、计量和报告的空间范围,企业应当对其本身发生的交易或者事项进行会计处理。确定会计主体是进行会计确认、计量和报告的基本前提。首先,会计进行确认和计量的交易或事项是以一个特定的会计主体为范围的t其次,企业在处理会计主体范围内的交易或事项时,是以这个会计主体为立场或从这个主体的角度出发的。
		
		比如,作为一个会计主体的某个企业向外赊购商品,会计的记录就是这个企业增加了商品和负债,而不是以供货方的角度进行记录。又比如,这个企业的一位所有者投入一台设备作企业的资本,会计就记为设备增加和资本增加,而这位所有者投在这个企业以外的其他财产就不记为该企业的资产和资本、他在企业之外的其他业务也不纳入企业的会计事项中。
		
		以上两点说明,会计主体概念是将会计所服务的单位作为一个独立的实体,将这个实体和与其相联系的人(所有者、管理人员、职工等)和单位(其他会计主体)区分开来,特别是要将会计主体与其所有者区分开来。也就是说,会计是为特定企业的交易或事项进行核算的,而不是为企业的某些个人事务进行核算的。试想:如果不这么严格确定会计主体,把企业和个人、其他单位的事务搅在一起,会计又如何为投资者、债权人等提供评价一个企业财务状况和经营成果的有用信息?
		
		要注意会计主体与法律主体的区别:
		\begin{enumerate}
			\item 一个会计主体可以不是法律主体。比如,某个人开一家商店,法律上并没有将这家独资企业与它的所有者区分开来。也就是说,如果商店欠了别人的债,商店的债权人不但可以要求用商店的财产偿还,也可以要求用业主个人的其他财产偿还。而在会计上只核算商店这一会计主体的资产和负债,尽管这些资产和负债在法律上也属于业主个人的。合伙企业的情况也与独资企业相似。另外,一个公司的分部或分支机构也可成为会计主体。

			\item 一个法律主体一般都是会计主体。而且,一个法律主体可以分成几个会计主体、几个法律主体也可以作为一个会计主体。比如,一个典型的法律主体——有限责任公司,就可以成为一个会计主体。由于在法律上公司也独立于它的所有者而存在,使得会计上区分会计主体和它的所有者比独资、合伙企业更容易。当然,如果独立的公司之间互相投资、一个公司成为其他公司的所有者,情形就会显得复杂。此时,也可以把这些公司组成的企业集团当作一个更大的会计主体。总之,会计主体不同于法律主体,会计主体的设立不是依据法律关系而是依据实质上的经济和管理关系。
		\end{enumerate}
		
		会计主体可以是企业,也可以是非营利组织,比如政府机关、事业单位、社会团体等等。本书主要以企业为例。
		
		\subsection{持续经营}

		持续经营,是指在可以预见的将来,企业将会按当前的规模和状态继续经营下去,不会停业,也不会大规模削减业务。在这一假设下,会计确认、计量和报告应当以企业持续、正常的生产经营活动为前提,即按“非清算状态”进行。
		
		这一假设是会计人员选择会计程序和方法的基础。就资产计价来说,持续经营下采用的方法与在清算时的方法是不同的。比如,正在生产线上加工的产品,如果企业持续经营,最终能够完工、出售,所以它们可以按其在加工过程中花费的实际成本(料、工、费)计价。但是,如果企业现在就停业、清算,这些没完工的产品就只能按当前的清算价值计量,其成本多少就没有意义了。因此,有了持续经营假设,就可以对这些资产采用实际成本计价。由此可见,持续经营假设为会计核算提供了一个正常的基础,保持了会计信息的连续性。当然,如果企业不能持续经营了,会计就要放弃这一假设,不再采用正常经营时的程序和方法。
		
		持续经营假设不仅是采用实际成本(即历史成本)计价的前提,还为考虑未来预期价值的其他计价方法提供了依据。另外,这一假设也有助于会计信息的相关性,因为许多有关企业的决策都是预期企业会继续存在的。
		
		值得注意的是,只要没有特别注明,企业提供的财务报告都是建立在持续经营假设下的。会计报表所反映的情况在持续经营假设下往往会与在清算状态下有所不同。
		
		\subsection{会计分期}

		会计分期,是指将一个企业持续经营的生产经营活动划分为一个个连续的、长短相同的期间,通过按期编报财务报告,及时向财务报告使用者提供有关企业财务状况、经营成果和现金流量的信息。
		
		当一个会计主体清算时,会计可以比较容易和准确地确定其从开业到停止营业的整个存在期间的全部情况和业绩。但由于绝大多数企业的存在时间往往都是好多年,会计不能只在企业结束营业时才一次性地“总算账”。会计需要为人们及时提供企业经营的信息,以便及时进行利益分配、明确责任、进行决策。所以,就有了定期提供财务报告的必要。
		
		每一期的财务报告反映的都是企业在当期的财务状况、现金流量和经营业绩等,各期财务报告应有明确的时间界限。但是,企业的各项实际活动都有自己的起讫时间,它们往往与会计分期的时间不相一致。因此,对于那些跨期的活动,会计就必须分清其对各会计期间的影响。由于有一些交易或事项在某个会计期间结束时还没有全部完成或实现,为了及时提供信息,各期的财务报告是在无法确知完整的未来情况下提出的,这会在一定程度上影响到财务报告的客观性和准确性。当然,有了持续经营假设,各会计期间对未来情况的预期是建立在持续经营假设基础上的。
		
		例如,企业销售产品,从与客户订立购销合同、并按合同将产品交付给客户、客户接受,直到从客户处收到货款并清账,整个过程也许并不能在一个会计期间完成。如果当客户接受企业的货物后,一个会计期间就结束了,客户付款时间已属于下一个会计期间。此时,会计就要确定产品销售所带来的收入应作为哪一期的收入、列入哪一期的财务报告中。而且,在采用某些确认方法下,在前一会计期间结束时,虽然还不能确定企业能否从客户处收回货款或者能收回多少,该期的财务报告中却应列示企业的这笔债权的确定数额。
		
		会计分期是会计核算的时间基础,它当然不会影响客观活动的连续进程。但是,它使会计信息在反映客观活动时带有了明显的时间特征。会计要素的确认与计量的一些原则,如权责发生制原则,是在会计分期的基础上产生的。
		
		会计期间通常为一年,也可称“会计年度”或“财政年度”。会计年度一般是日历年度,但也有以企业的自然营业年度为会计年度的,常常从营业量最少的时间为会计年度的起点。我国是以日历年度为会计年度。为及时提供会计信息,还可以将会计年度再划分为半年度、季度和月度。半年度、季度和月度这些短于一个会计年度的报告期间,均称为会计中期。

		\subsection{货币计量}
		
		货币计量,是指会计主体在会计信息的处理和报告时以货币作为计量尺度。
		
		可以作为计量尺度的单位有很多。就企业所拥有的各种资源来说,可以采用的计量单位就有重量、长度、面积等多种。比如,某企业拥有$1,000$公斤原材料、两幢$20,000$平方米的厂房、三辆$20$吨运输用的汽车等等。但是,采用这些计量单位无法体现上述资源的共性、无法在量上将它们进行比较和汇总。货币是商品的一般等价物,是衡量一般商品价值的共同尺度。采用货币计量就可将企业所拥有资源的价值用一个共同的标准表现出来,并使它们有了比较和汇总的基础。不过货币计量使会计无法直接记录和报告诸如企业员工的健康和工作经验、人际关系、设备的性能和效率、企业组织结构等情况。也就是说,货币计量的范围是受到明显的限制的。
		
		以货币作为计量尺度,正是会计信息的重要特征。在这一特征下,还需要有一个重要的条件或设定,这就是币值稳定。作为计量单位自身应保持不变,比如一米的长度、一公斤的重量都是始终不变的,否则就无法以此衡量和比较物体的长度和重量。同样地,会计在采用货币计时,就认为货币自身的价值——购买力(可用一般物价指数表示)不会随时间而改变。也就是说,会计上不区分不同时期货币购买力的变化、不考虑通货膨胀(币值下降)等的影响。比如,企业分别在一般物价指数为$100$和$150$时同样用$20,000$元购买了资产,这两次购买的资产价值并不相等,因为货币的购买力不同。但是会计上却将在不同购买力下的这两笔$20,000$元一视同仁,加总后以$40,000$元资产予以记录和报告,似乎两者的币值没有变动。所以,如果发生持续时间长、严重通货膨胀,币值稳定的假设可能导致资产价值的严重失实等问题。此时,就有了对币值稳定假设进行修正的必要。
		
		采用货币计量还有一个实际的操作问题:世界上作为流通手段的币种很多,如人民币、美元、日元、欧元等等。实务中,会计究竟用哪一种货币进行日常记账呢?也就是用什么货币作为“记账本位币”呢?我国企业一般以人民币为记账本位币。以人民币为记账本位币时,企业如果发生外币业务,就应将外币换算为人民币。我国《企业会计准则第19号——外币折算》规定:“企业通常应选择人民币作为记账本位币。业务收支以人民币以外的货币为主的企业,可以按照本准则第五条规定选定其中一种货币作为记账本位币。但是,编报的财务报表应当折算为人民币。”

	\section{资产、负债和所有者权益}

	会计基本假设为会计要素的计量、确认和报告提供了前提条件。会计要素是根据 交易或者亊项的经济特征所确定的财务会计对象的基本分类,它既是会计确认和计量的依据,也是确定财务报表结构和内容的基础。资产和权益,是最基本的会计要素。

		\subsection{资产}
			资产是指企业过去的交易或者事项形成的、由企业拥有或者控制的、预期会给企业带来经济利益的资源。

			从以上定义可知,资产是一种经济资源,这种经济资源必须同时具备以下基本特征。
			\begin{enumerate}
				\item[(一)] 预期会给企业带来经济利益 \\
					这是经济资源的必备条件。“预期会给企业带来经济利益”,是指直接或间接导致现金或现金等价物流人的企业的潜力。这就是说资产能够使企业在将来有现金流入而能在将来带来现金流人的资源的条件是:

					\begin{enumerate}
						\item[1、] 它们本身就是现金,企业可用它们获得其他资源;
						\item[2、] 它们可以转化为现金,如债权或股权,企业将来可以通过向被投资方收回本金和取得利息或红利而得到现金;
						\item[3、] 它们是最终准备出售的货物,企业将来可以通过出售它们而收取现金;
						\item[4、] 它们可以用于未来的经营活动,如企业的经营场所、经营设施和其他经营条件,企业可通过使用它们而在未来获得现金流入。
					\end{enumerate}

					由此可见,已报废的生产设备、收不回来的货款、库存的已毁损的商品等,由于它们不能再给企业带来现金流入,即不会再提供经济利益,而不能作为企业的资产。

				\item[(二)] 由企业所拥有或控制 \\
					作为资产的经济资源必须是由企业(即一个会计主体)所拥有或控制的。所谓“由企业拥有或者控制”,是指企业享有某项资源的所有权,或者虽然不享有某项资源的所有权,但该项资源被企业所控制。资产的这一特征表明了两点:一是,为企业所利用的经济资源并不都是企业的资产;二是,企业的资产其所有权有可能不属于企业。比如,企业有一只游船,经营假日水上旅馆。该船所经过的江河,尽管是很有价值的资源,但那不是企业的资产,因为江河不由企业所拥有或控制。至于那只船,如果企业己获得其所有权,就是企业的资产;如果是企业租入的,这就要视具体的租赁合同而定。假设租赁合同规定租期很长,与游船未来的使用寿命相当,或者租金总额折算起来又与船的价格接近等等,总之能够表明游船在其使用寿命期所带来的收益基本都归企业所有,而船的贬值、毁损等所带来的损失实际上也由企业负担。那么,这只船尽管只是租的,法律上的所有权不属于企业,但是这船实质上在租赁期间已由企业所控制,这船也应作为企业的资产。将企业不拥有所有权、以融资租入方式获得的资源作为资产,体现了会计的“经济实质重于法律形式”的原则。
				\item[(三)] 由过去的交易和事项形成 \\
					资产应该是由过去的交易或事项形成的。所谓“交易或者事项”,包括购买、生产、建造行为或其他交易或者事项。预期在未来发生的交易或者事项是不会形成资产的。比如,前面提到的那只游船,它是企业过去已经购入或融资租入的,这才成为该企业的资产;假如企业只是有购买游船的打算,或者已经与游船制造商进行了谈判,由于企业还没有现实地持有那只船,船就不是该企业的资产。
					
					从资产的定义和特征,可以说明企业资产的过去、现在和未来之间的关系。企业现在所持有的资产,是企业过去通过交易或事项、付出一定的代价取得的,它在未来有经济利益作为回报。所以,资产是一个会计主体的“过去的付出和未来的回报”。
			\end{enumerate}

		\subsection{权益}
			与资产相对应的是权益。权益是指资产的所有者对资产所提出的要求权(或索偿 权、主张权)。从一个会计主体的角度看,权益是会计主体承担的未来应向资产所有者 提供经济利益的一种责任或义务。权益也反映了一个会计主体的资金来源。具体包 括:债权人权益(即负债)和所有者权益(即资本,包括投人资本和滋生资本)。

			\begin{enumerate}
				\item[(一)] 负债 \\
					负债有以下几个基本特征:
					\begin{enumerate}
						\item 由过去的交易或事项形成的现时义务。
							负债是企业的现时义务,所谓“现时义务”是指企业在现行条件下已承担的义务。未来发生的交易或者事项形成的义务,不属于现时义务,不应当确认为负债。比如,企业赊购货物就形成了一项应付账款,企业向银行借款也形成了一项负债。这购货与借款都已实际发生。如果企业计划向银行贷款,就不形成企业的负债。
						\item 预期会导致经济利益流出企业。
							负债在将来某一时日偿还时,会使企业流出经济利益。经济利益是由资产带来的或者是由提供劳务获得的,所以,经济利益的流出就是放弃资产或提供劳务。也就是说,企业在未来履行义务时,将支付现金、转让其他资产或提供劳务。当然,也可以用一项义务或责任,代另一项义务,不过这只是义务或责任的向后延期。

							负债一般都有一个确定的偿付金额,即使没有一个确定的金额,至少也是能够合理估计的。比如,企业应交所得税,在经税务部门核定、实际缴纳之前是会计按税法规定估计的。而企业所欠的银行借款的本金和利息其金额一般都是确定的。

							除此之外,负债往往有一个确定的债权人(债主)和偿付日期,而且往往是一项带有强制性的义务。但是,负债有时可能并不具备这两个条件。比如,企业对自己出售的商品作出了产品质量保证,产品有质量问题时企业承担产品的维修、退换所造成的损失,这个义务发生时,债权人、偿付日期和金额都是不确定的。会计上出于谨慎的考虑,当因这种保证所带来的义务在未来偿付的可能性较大时,可以作为一项负债。
					\end{enumerate}
				\item[(二)] 所有者权益 \\
					所有者权益是指企业资产扣除负债后由所有者享有的剩余权益。
					
					所有者权益也可以被看成是所有者对企业净资产的要求权,是除了负债之外会计主体的另一项责任或义务。不过,所有者权益不同于负债,它没有清偿的期限。而且它仅仅是在持续经营假设下的会计概念,并不表示企业在实际清算时所有者在法律上的索偿权。因为在持续经营下所有者权益主要是以成本而不是按清算价值计量,所以不能体现实际的索偿要求;而且索偿权只是在实际清算时才有意义,在持续经营下就没有实际的意义。
					
					所有者权益主要包括两大部分,第一部分是投入资本,即所有者直接投入企业的资本;第二部分是留存利润,是指企业主要在日常活动中所赚取的、并仍留在企业而没有以红利等形式支付给所有者的利润。除此之外,属于所有者权益的有时还有其他内容,比如企业在非日常活动中取得的利得和损失。有关所有者权益的具体内容我们将在后面各章中详细介绍。
			\end{enumerate}

	\section{会计等式}
		\subsection{会计等式}
		资产和权益,实际上是同一个事物的两个不同方面的表现。资产是经济资源,而权益表明了企业因掌握这些经济资源而承担的经济责任;权益是企业的资金来源,而资产反映了资金运用。
		
		比如,某个会计主体有如下资源:

		\begin{enumerate}
			\item 现金:1元;
			\item 商品:4元;
			\item 设备:5元。
		\end{enumerate}
		
		总共10元。
		
		这10元,从另一角度看,分别属于:
		
		\begin{enumerate}
			\item 负债:3元;
			\item 所有者权益:7元。
		\end{enumerate}
		
		由此可见,资产和权益是从货币计量角度反映的同一事物的两面而已。因此,资产和权益存在这样一种数量关系:

		$$\text{资产}=\text{权益}$$

		或者
		
		$$\text{资产} = \text{负债} + \text{所有者权益}$$
		
		以上式子称为“会计等式”或“会计方程式”。无论资产和权益如何变化,等式总是成立的。如果企业要取得资产,就必须筹集与这些资产金额相等的资金;反过来,企业筹集的资金总是同时体现为各种形态的资产。所以,有多少金额的资产就有多少金额的权益,有多少金额的权益就有多少金额的资产,两者的总额必然相等。会计等式是会计报表的基础,也是复式记账的基础。
		
		上述等式是会计的基本等式,基本等式可以变化和扩展。先看变化。比如,可以用基本等式反映所有者权益是资产减去负债后的余额,是所有者对企业净资产的索偿权:%$$\eqalign{\text{资产} - \text{负债} &= \text{所有者权益} \cr \text{净资产} &= \text{所有者权益}\cr}$$
		\begin{align*}
			\text{资产} - \text{负债} &= \text{所有者权益} \\ 
			\text{净资产} &= \text{所有者权益}
		\end{align*}
		
		会计等式的扩展是在收入和费用概念引入之后,见本章后文。

		\subsection{会计事项}

		尽管一个会计主体的资产和权益是衡等的,但是资产和权益又是处在不断的变化之中的。资产和权益的变化是由企业的经济活动引起的。因此,会计上把引起资产和权益变化的企业经济活动称为“会计事项”或“经济业务”(即前文提到的“交易或事项”),会计事项是会计确认和计量的直接对象。
		
		\subsubsection{会计事项分析}
		
		有了会计等式概念,我们会发现:一个会计主体的每一笔会计事项都会对会计记录产生双重影响,或者说,会计对每一笔金额变动都看两面。以例\ref{example:2-1}进行分析。

		\example\label{example:2-1} 张三于$20 \times 8$年$12$月$20$日注册了一家商店。张三以$10,000$元现金作资本。商店于$20 \times 9$年$1$月份开始营业。以下是$1$月份的会计事项及其分析:

		(1)商店用现金向甲企业购入$4,000$元商品。在此项业务之前,商店已有开业时张三投入的$10,000$元:是所有者权益(实收资本)和资产(现金)同时增加。从$1$月份的这第一笔会计事项看:采购业务在使商品增加的同时,又使现金减少。这一会计事项没有影响资产和权益的总额,但使资产形态或结构发生了变化。

		$$\frac{\text{资产}}{\text{现金} \quad \text{商品} \quad \text{用具} \quad \text{应收账款}} = \frac{\text{负债}}{\text{应付账款} \quad \text{银行借款}} + \frac{\text{所有者权益}}{\text{实收资本} \quad \text{留存利润}}$$

		(2)张三将自己的物品$2,000$元投资给商店作为营业用具。这一会计事项是张三将会计主体之外的物品投入,使商店在增加所有者权益的同时,又增加了资产,使会计等式两边同时增加了$2,000$元。

		(3)商店向乙企业赊购$7,000$元商品。赊购使商品增加的同时,又增加了负债。

		(4)商店用现金归还前欠乙企业货款$1,500$元。(4)与(3)正好相反,是资产(现金)与权益(负债)同时减少。

		(5)张三从商店提取$800$元现金用于自己个人开支。这一会计事项因张三提款而使商店减少了实收资本和现金,使等式两边同时喊少$800$元。

		(6)商店向银行借款直接偿还乙企业货款$1,000$元。此会计事项说明,商店负债的结构发生了变化,从原来欠乙企业的变成了现在欠银行的。负债的一增一减,不影响等式两边的总额。

		(7)张三将自己欠银行的贷款$1,200$元转给商店,今后由商店归还,作为张三投资减少。这一会计事项使商店的负债增加,同时又使所有者权益减少,这是等式右边权益结构的变化,不影响总额。会计主体假设使张三个人事务与商店业务能够严格区分。

		(8)张三准备与李四合伙经营商店,李四先替商店偿还乙企业货款$2,000$元,作为投资。与(7)相反,此业务使商店权益结构发生了另一种变化:商店在负债减少的同时增加了所有者权益。商店由独资企业变为合伙企业。

		(9)张三将投资中的$4,000$元转让给李四。此业务仅仅使商店的所有者权益结构发生变化,即实收资本一增一减。

		(10)商店零售$6,000$元商品,售价共$7,800$元,已收到现金。从此业务看:一方面,商店减少了商品$6,000$元,同时又增加现金$7,800$元,资产净增加$1,800$元;另一方面,这净增的$1,800$元是商店赚来的,属于所有者权益。但这不是所有者的直接投入而是“留存利润”。

		(11)商店向丙单位销售$4,000$元商品,售价为$5,200$元,尚未收到现金。这一会计事项与(10)相似,但这次里赊销。所以,没有增加现金,而是增加了“应收账款”。

		(12)商店支付$1$月份营业用房的房租$700$元。此业务使现金减少了,同时房租是为销售商品取得收入而必须的花费,另一方面又减少了所有者权益(留存利润)。

		(13)商店支付店员当月工资$600$元。与(12)一样,本业务也使商店在减少现金的同时减少了留存利润。

		\subsubsection{会计事项类型}

		从以上会计事项分析可知:每一笔会计事项都涉及两个方面。而且,资产和权益总额始终保持相等。会计事项引起资产和权益的变动共有表2-1所列几种类型。








